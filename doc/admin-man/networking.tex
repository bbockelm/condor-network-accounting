%%%%%%%%%%%%%%%%%%%%%%%%%%%%%%%%%%%%%%%%%%%%%%%%%%%%%%%%%%%%%%%%%%%%%%
\section{\label{sec:Networking}Networking (includes sections on Port Usage and CCB)}
%%%%%%%%%%%%%%%%%%%%%%%%%%%%%%%%%%%%%%%%%%%%%%%%%%%%%%%%%%%%%%%%%%%%%%
\index{network}

This section on
network communication in Condor
discusses which network ports are used,
how Condor behaves on machines with multiple network interfaces
and IP addresses,
and how to facilitate functionality in a pool that spans
firewalls and private networks.

The security section of the manual contains some
information that is relevant to the discussion of network
communication which will not be duplicated here, so please
see section~\ref{sec:Security} as well.

Firewalls, private networks, and network address translation (NAT)
pose special problems for Condor.
There are currently two main mechanisms for dealing with firewalls
within Condor:

\begin{enumerate}

\item Restrict Condor to use a specific range of port numbers, and
  allow connections through the firewall that use any port within the
  range.

\item Use \Term{Condor Connection Brokering} (CCB).

\end{enumerate}

Each method has its own advantages and disadvantages,
as described below.


%%%%%%%%%%%%%%%%%%%%%%%%%%%%%%%%%%%%%%%%%%%%%%%%%%%%%%%%%%%%%%%%%%%%%%
% all of these define their own \subsection, so include them directly
%%%%%%%%%%%%%%%%%%%%%%%%%%%%%%%%%%%%%%%%%%%%%%%%%%%%%%%%%%%%%%%%%%%%%%

%%%%%%%%%%%%%%%%%%%%%%%%%%%%%%%%%%%%%%%%%%%%%%%%%%%%%%%%%%%%%%%%%%%%%%%%%%%
\subsection{\label{sec:Port-Details}Port Usage in Condor}
%%%%%%%%%%%%%%%%%%%%%%%%%%%%%%%%%%%%%%%%%%%%%%%%%%%%%%%%%%%%%%%%%%%%%%%%%%
\index{port usage}

%%%%%%%%%%%%%%%%%%%%%%%%%%%%%%%%%%%%%%%%%%%%%%%%%%%%%%%%%%%%%%%%%%%%%%%%%%%
\subsubsection{\label{sec:IPv4-Port-Specification}IPv4 Port Specification}
%%%%%%%%%%%%%%%%%%%%%%%%%%%%%%%%%%%%%%%%%%%%%%%%%%%%%%%%%%%%%%%%%%%%%%%%%%%
\index{IPv4 port specification}
\index{port usage!IPv4 port specification}
The general form for IPv4 port specification is 
\begin{verbatim}
<IP:port?param1name=value1&param2name=value2&param3name=value3&...> 
\end{verbatim}

These parameters and values are URL-encoded.
This means any special character is encoded with \Percent,
followed by two hexadecimal digits specifying the ASCII value.
Special characters are any non-alphanumeric character. 

Condor currently recognizes the following parameters with an
IPv4 port specification:
\begin{description}
\item[\Expr{CCBID}]
  Provides contact information for forming a CCB connection to a daemon, 
  or a space separated list, if the daemon is registered with more than 
  one CCB server.
  Each contact information is specified in the form of \verb|IP:port#ID|.
  Note that spaces between list items will be URL encoded by \verb|%20|.
\item[\Expr{PrivNet}]
  Provides the name of the daemon's private network.
  This value is specified in the configuration with 
  \MacroNI{PRIVATE\_NETWORK\_NAME}.
\item[\Expr{sock}]
  Provides the name of \Condor{shared\_port} daemon named socket.
\item[\Expr{PrivAddr}]
  Provides the daemon's private address in form of
  \Expr{IP:port}.
\end{description}

%%%%%%%%%%%%%%%%%%%%%%%%%%%%%%%%%%%%%%%%%%%%%%%%%%%%%%%%%%%%%%%%%%%%%%%%%%%
\subsubsection{\label{sec:Ports-Standard}Default Port Usage}
%%%%%%%%%%%%%%%%%%%%%%%%%%%%%%%%%%%%%%%%%%%%%%%%%%%%%%%%%%%%%%%%%%%%%%%%%%%

Every Condor daemon listens on a network port for incoming commands.
(Using \Condor{shared\_port}, this port may be shared between multiple
daemons.)
Most daemons listen on a dynamically assigned port.
In order to send a message,
Condor daemons and tools locate the correct port to use
by querying the \Condor{collector},
extracting the port number from the ClassAd.
One of the attributes included in every daemon's ClassAd is the full
IP address and port number upon which the daemon is listening.

To access the \Condor{collector} itself,
all Condor daemons and tools
must know the port number  where the \Condor{collector} is listening.
The \Condor{collector} is the only daemon with a well-known,
fixed port.
By default, Condor uses port 9618 for the \Condor{collector} daemon.
However, this port number can be changed (see below).

As an optimization for daemons and tools communicating with another
daemon that is running on the same host,
each Condor daemon can be configured to
write its IP address and port number into a well-known file.
The file names are controlled using the \MacroB{<SUBSYS>\_ADDRESS\_FILE}
configuration variables,
as described in section~\ref{param:SubsysAddressFile} on
page~\pageref{param:SubsysAddressFile}. 

\Note In the 6.6 stable series, and Condor versions earlier than
6.7.5, the \Condor{negotiator} also listened on a fixed, well-known
port (the default was 9614).
However, beginning with version 6.7.5, the \Condor{negotiator} behaves
like all other Condor daemons, and publishes its own ClassAd to the
\Condor{collector} which includes the dynamically assigned port 
the \Condor{negotiator} is listening on.
All Condor tools and daemons that need to communicate with the
\Condor{negotiator} will either use the
\Macro{NEGOTIATOR\_ADDRESS\_FILE} or will query the
\Condor{collector} for the \Condor{negotiator}'s ClassAd.

Sites that configure any checkpoint servers will introduce
other fixed ports into their network.
Each \Condor{ckpt\_server} will listen to 4 fixed ports: 5651, 5652,
5653, and 5654.
There is currently no way to configure alternative values for any of
these ports.


%%%%%%%%%%%%%%%%%%%%%%%%%%%%%%%%%%%%%%%%%%%%%%%%%%%%%%%%%%%%%%%%%%%%%%%%%%%
\subsubsection{\label{sec:Ports-NonStandard}Using 
a Non Standard, Fixed Port for the \Condor{collector}}
%%%%%%%%%%%%%%%%%%%%%%%%%%%%%%%%%%%%%%%%%%%%%%%%%%%%%%%%%%%%%%%%%%%%%%%%%%%
\index{port usage!nonstandard ports for central managers}
By default,
Condor uses port 9618 for the \Condor{collector} daemon.
To use a different port number for this daemon,
the configuration variables that tell Condor these communication
details are modified.
Instead of
\begin{verbatim}
CONDOR_HOST = machX.cs.wisc.edu
COLLECTOR_HOST = $(CONDOR_HOST)
\end{verbatim}
the configuration might be
\begin{verbatim}
CONDOR_HOST = machX.cs.wisc.edu
COLLECTOR_HOST = $(CONDOR_HOST):9650
\end{verbatim}

If a non standard port is defined, the same value of
\MacroNI{COLLECTOR\_HOST} (including the port) must be used for all
machines in the Condor pool.
Therefore, this setting should be modified in the global
configuration file (\File{condor\_config} file),
or the value must be duplicated across
all configuration files in the pool if a single configuration file
is not being shared.

When querying the \Condor{collector} for a remote pool that is running
on a non standard port, any Condor tool that accepts the \Opt{-pool}
argument can optionally be given a port number.  For example:
\footnotesize
\begin{verbatim}
        % condor_status -pool foo.bar.org:1234
\end{verbatim}
\normalsize


%%%%%%%%%%%%%%%%%%%%%%%%%%%%%%%%%%%%%%%%%%%%%%%%%%%%%%%%%%%%%%%%%%%%%%%%%%%
\subsubsection{\label{sec:Ports-Dynamic-Collector}Using 
a Dynamically Assigned Port for the \Condor{collector}}
%%%%%%%%%%%%%%%%%%%%%%%%%%%%%%%%%%%%%%%%%%%%%%%%%%%%%%%%%%%%%%%%%%%%%%%%%%%

On single machine pools, 
it is permitted to configure the
\Condor{collector} daemon
to use a dynamically assigned port,
as given out by the operating system.
This prevents port conflicts with other services on the same machine.
However, a dynamically assigned port is only to be used on
single machine Condor pools,
and only if the
\Macro{COLLECTOR\_ADDRESS\_FILE} 
configuration variable has also been defined.
This mechanism allows all of the Condor daemons and tools running on
the same machine to find the port upon which the \Condor{collector}
daemon is listening,
even when this port is not defined in the
configuration file and is not known in advance.

To enable the \Condor{collector} daemon to use a dynamically assigned port,
the port number is set to 0 in the \Macro{COLLECTOR\_HOST}
variable.
The \MacroNI{COLLECTOR\_ADDRESS\_FILE}
configuration variable must also be defined,
as it provides a known file where the IP address
and port information will be stored.
All Condor clients know to look at the
information stored in this file.
For example:
\footnotesize
\begin{verbatim}
COLLECTOR_HOST = $(CONDOR_HOST):0
COLLECTOR_ADDRESS_FILE = $(LOG)/.collector_address
\end{verbatim}
\normalsize

\Note Using a port of 0 for the \Condor{collector}
and specifying a
\MacroNI{COLLECTOR\_ADDRESS\_FILE}
only works in Condor version 6.6.8 or later in the 6.6 stable series,
and in version 6.7.4 or later in the 6.7 development series.
Do not attempt to do this with older versions of Condor.

Configuration definition of \MacroNI{COLLECTOR\_ADDRESS\_FILE}
is in section~\ref{param:SubsysAddressFile} on
page~\pageref{param:SubsysAddressFile},
and
\MacroNI{COLLECTOR\_HOST}
is in
section~\ref{param:CollectorHost} on
page~\pageref{param:CollectorHost}.


%%%%%%%%%%%%%%%%%%%%%%%%%%%%%%%%%%%%%%%%%%%%%%%%%%%%%%%%%%%%%%%%%%%%%%%%%%%
\subsubsection{\label{sec:Ports-Firewalls}Restricting Port Usage to
 Operate with Firewalls}
%%%%%%%%%%%%%%%%%%%%%%%%%%%%%%%%%%%%%%%%%%%%%%%%%%%%%%%%%%%%%%%%%%%%%%%%%%%

\index{port usage!firewalls}
If a Condor pool is completely behind a firewall,
then no special consideration or port usage is needed.
However, if there is a firewall between the machines within
a Condor pool, then
configuration variables may be set to force the usage of
specific ports, and to utilize a specific range of ports.

By default,
Condor uses port 9618 for the \Condor{collector} daemon,
and dynamic (apparently random) ports for everything else.
See section~\ref{sec:Ports-Dynamic-Collector},
if a dynamically assigned port is desired for the
\Condor{collector} daemon.

All of the Condor daemons on a machine
may be configured to share a single
port.  See section~\ref{sec:Config-shared-port} for more information.

The configuration variables
\Macro{HIGHPORT} and \Macro{LOWPORT} facilitate setting a restricted
range of ports that Condor will use.
This may be useful when some machines are behind a firewall.
The configuration macros
\MacroNI{HIGHPORT} and \MacroNI{LOWPORT} 
will restrict dynamic ports to the range specified.
The configuration variables are fully defined
in section~\ref{sec:Network-Related-Config-File-Entries}.
All of these ports must be greater than 0 and less than 65,536.
Note that both \MacroNI{HIGHPORT} and \MacroNI{LOWPORT} must be at 
least 1024 for Condor version 6.6.8.
In general, use ports greater than 1024,
in order
to avoid port conflicts with standard services on the machine.
Another reason for using ports greater than 1024 is that
daemons and tools are often not run as \Login{root},
and only \Login{root} may listen to a port lower than 1024.
Also, the range must include enough ports that are not in use, 
or Condor cannot work.

The range of ports assigned may be restricted based on 
incoming (listening) and outgoing (connect) ports
with the configuration variables
\Macro{IN\_HIGHPORT},
\Macro{IN\_LOWPORT},
\Macro{OUT\_HIGHPORT}, and
\Macro{OUT\_LOWPORT}.
See section~\ref{sec:Network-Related-Config-File-Entries}
for complete definitions of these configuration variables.
A range of ports lower than 1024 for daemons
running as \Login{root} is appropriate for incoming ports,
but not for outgoing ports.
The use of ports below 1024 (versus above 1024)
has security implications; 
therefore, it is inappropriate to assign a range that crosses
the 1024 boundary.


\Note Setting \MacroNI{HIGHPORT} and \MacroNI{LOWPORT} will not
automatically force the \Condor{collector} to bind to a port within
the range.
The only way to control what port the \Condor{collector} uses is by
setting the \MacroNI{COLLECTOR\_HOST} (as described above).

The total number of ports needed depends on the size of the pool,
the usage of the machines within the pool (which machines
run which daemons),
and the number of jobs that may execute at one time.
Here we discuss how many ports are used by each
participant in the system.  This assumes that \Condor{shared\_port}
is not being used.  If it \emph{is} being used, then all daemons
can share a single incoming port.

The central manager of the pool needs
\Expr{5 + \MacroNI{NEGOTIATOR\_SOCKET\_CACHE\_SIZE}}
ports for daemon communication,
where 
\Macro{NEGOTIATOR\_SOCKET\_CACHE\_SIZE}
is specified in the
configuration or defaults to the value 16.

Each execute machine (those machines running a \Condor{startd} daemon)
requires
\Expr{ 5 + (5 * number of slots advertised by that machine)}
ports.
By default, the number of slots advertised
will equal the number of physical CPUs in that machine.

Submit machines (those machines running a \Condor{schedd} daemon)
require
\Expr{ 5 + (5 *  \MacroNI{MAX\_JOBS\_RUNNING})} ports.
The configuration variable \Macro{MAX\_JOBS\_RUNNING}
limits (on a per-machine basis, if desired)
the maximum number of jobs.
Without this configuration macro,
the maximum number of jobs that could be simultaneously
executing at one time
is a function of the number of reachable execute machines. 

Also be aware that \MacroNI{HIGHPORT} and \MacroNI{LOWPORT}
only impact dynamic port selection used by the Condor system,
and they do not impact port selection used by jobs submitted to Condor.
Thus, jobs submitted to Condor that may create
network connections may not work in a port restricted environment.
For this reason, specifying \MacroNI{HIGHPORT} and \MacroNI{LOWPORT}
is not going to produce the
expected results if a user submits MPI applications to be executed under
the parallel universe.

Where desired, a local
configuration for machines \emph{not} behind a firewall
can override the usage of \MacroNI{HIGHPORT} and \MacroNI{LOWPORT},
such that the ports used for these machines are not restricted.
This can be accomplished by adding the following to the
local configuration file of those machines \emph{not}
behind a firewall:
\begin{verbatim}
HIGHPORT = UNDEFINED
LOWPORT  = UNDEFINED
\end{verbatim}


If the maximum number of ports allocated using 
\MacroNI{HIGHPORT} and \MacroNI{LOWPORT}
is too few,
socket binding errors of the form
\footnotesize
\begin{verbatim}
failed to bind any port within <$LOWPORT> - <$HIGHPORT>
\end{verbatim}
\normalsize
are likely to appear repeatedly in log files.


%%%%%%%%%%%%%%%%%%%%%%%%%%%%%%%%%%%%%%%%%%%%%%%%%%%%%%%%%%%%%%%%%%%%%%%%%%%
\subsubsection{\label{sec:Ports-MultipleCollectors}Multiple Collectors}
%%%%%%%%%%%%%%%%%%%%%%%%%%%%%%%%%%%%%%%%%%%%%%%%%%%%%%%%%%%%%%%%%%%%%%%%%%%
\index{port usage!multiple collectors}
\Todo


%%%%%%%%%%%%%%%%%%%%%%%%%%%%%%%%%%%%%%%%%%%%%%%%%%%%%%%%%%%%%%%%%%%%%%%%%%%
\subsubsection{\label{sec:Ports-Conflicts}Port Conflicts}
%%%%%%%%%%%%%%%%%%%%%%%%%%%%%%%%%%%%%%%%%%%%%%%%%%%%%%%%%%%%%%%%%%%%%%%%%%%
\index{port usage!conflicts}
\Todo



\input{admin-man/shared-port-daemon.tex}

\input{admin-man/multiple-interfaces.tex}

%%%%%%%%%%%%%%%%%%%%%%%%%%%%%%%%%%%%%%%%%%%%%%%%%%%%%%%%%%%%%%%%%%%%%%
\subsection{\label{sec:CCB}Condor Connection Brokering (CCB)}
%%%%%%%%%%%%%%%%%%%%%%%%%%%%%%%%%%%%%%%%%%%%%%%%%%%%%%%%%%%%%%%%%%%%%%
\index{CCB (Condor Connection Brokering)}

Condor Connection Brokering, or CCB, is a way of allowing Condor
components to communicate with each other when one side is in a
private network or behind a firewall.  Specifically, CCB allows
communication across a private network boundary in the following
scenario: a Condor tool or daemon (process A) needs to connect to a
Condor daemon (process B), but the network does not allow a TCP
connection to be created from A to B; it only allows connections from
B to A.  In this case, B may be configured
to register itself with a CCB server that both A and B can connect to.
Then when A needs to connect to B, it can send a request to the CCB
server, which will instruct B to connect to A so that the two can
communicate.

As an example, consider a Condor execute node that is within
a private network. 
This execute node's \Condor{startd} is process B.
This execute node cannot normally run jobs submitted from a machine
that is outside of that private network, 
because bi-directional connectivity between the submit node and the
execute node is normally required.  
However, 
if both execute and submit machine can connect to the CCB server,
if both are authorized by the CCB server,
and if it is possible for the execute node within the private network
to connect to the submit node,
then it is possible for the submit node to run jobs on the
execute node.

To effect this CCB solution,
the execute node's \Condor{startd} within the private network
registers itself with the CCB
server by setting the configuration variable \Macro{CCB\_ADDRESS}.
The submit node's \Condor{schedd} communicates with the CCB server,
requesting that the execute node's \Condor{startd} open the TCP
connection.
The CCB server forwards this request to the execute node's \Condor{startd},
which opens the TCP connection.
Once the connection is open, bi-directional communication is enabled.

If the location of the execute and submit nodes is reversed 
with respect to the private network,
the same idea applies:
the submit node within the private network registers itself with a CCB server,
such that when a job is running and the execute node needs to connect back to
the submit node (for example, to transfer output files), 
the execute node can connect by going through CCB to request a connection.

If both A and B are in separate private networks, then CCB alone
cannot provide connectivity.  However, if an incoming port or port
range can be opened in one of the private networks, then the situation
becomes equivalent to one of the scenarios described above and CCB can
provide bi-directional communication given only one-directional
connectivity.  See section~\label{sec:Port-Details} for information on
opening port ranges.  Also note that CCB works nicely with
\Condor{shared\_port}.

Unfortunately at this time, CCB does not support standard universe jobs.

Any \Condor{collector} may be used as a CCB server.  There is no
requirement that the \Condor{collector} acting as the CCB server
be the same \Condor{collector} that a daemon
advertises itself to (as with \MacroNI{COLLECTOR\_HOST}).
However, this is often a convenient choice.

\subsubsection{Example Configuration}

This example assumes that there is a pool of machines in a private
network that need to be made accessible from the outside,
and that the \Condor{collector} (and therefore CCB server)
used by these machines is accessible from the outside.
Accessibility might be achieved by
a special firewall rule for the \Condor{collector} port,
or by being on a dual-homed machine in both networks.

The configuration of variable \MacroNI{CCB\_ADDRESS} on
machines in the private network causes registration with
the CCB server as in the example:

\begin{verbatim}
  CCB_ADDRESS = $(COLLECTOR_HOST)
  PRIVATE_NETWORK_NAME = cs.wisc.edu
\end{verbatim}

The definition of \MacroNI{PRIVATE\_NETWORK\_NAME} ensures that all
communication between nodes within the private network continues to happen
as normal, and without going through the CCB server.
The name chosen for \MacroNI{PRIVATE\_NETWORK\_NAME} should be different
from the private network name chosen for any Condor installations that
will be communicating with this pool.

Under Unix, and with large Condor pools,
it is also necessary to give the \Condor{collector} acting as the CCB server
a large enough limit of file descriptors.
This may be accomplished with the configuration variable
\Macro{MAX\_FILE\_DESCRIPTORS} or an equivalent.
Each Condor process configured to use CCB with \MacroNI{CCB\_ADDRESS}
requires one persistent TCP connection to the CCB server.
A typical execute node
requires one connection for the \Condor{master},
one for the \Condor{startd},
and one for each running job, as represented by a \Condor{starter}.
A typical submit machine
requires one connection for the \Condor{master},
one for the \Condor{schedd},
and one for each running job, as represented by a \Condor{shadow}.
If there will be no administrative commands required
to be sent to the \Condor{master} from outside of
the private network, then CCB may be disabled in the \Condor{master}
by assigning \MacroNI{MASTER.CCB\_ADDRESS} to nothing:
\begin{verbatim}
  MASTER.CCB_ADDRESS =
\end{verbatim}

Completing the count of TCP connections in this example:
suppose the pool consists of 500 8-slot
execute nodes and CCB is not disabled in the configuration of the
\Condor{master} processes.
In this case, the count of needed file descriptors plus some extra
for other transient connections to the collector is
500*(1+1+8)=5000.
Be generous, and give it twice as many
descriptors as needed by CCB alone:

\begin{verbatim}
  COLLECTOR.MAX_FILE_DESCRIPTORS = 10000
\end{verbatim}

\subsubsection{Security and CCB}

The CCB server authorizes all daemons that register themselves with it
(using \Macro{CCB\_ADDRESS}) at the DAEMON authorization level (these
are playing the role of process A in the above description).  It
authorizes all connection requests (from process B) at the READ
authorization level.  As usual, whether process B authorizes process A
to do whatever it is trying to do is up to the security policy for
process B; from the Condor security model's point of view, it is as if
process A connected to process B, even though at the network layer,
the reverse is true.

\subsubsection{Troubleshooting CCB}

Errors registering with CCB or requesting connections via CCB are
logged at level \Dflag{ALWAYS} in the debugging log.
These errors may be identified by searching for "CCB" in the log message.
Command-line tools require the argument
\Opt{-debug} for this information to be visible.  To see details of
the CCB protocol add \Dflag{FULLDEBUG} to the debugging options for
the particular Condor subsystem of interest.
Or, add \Dflag{FULLDEBUG} to
\MacroNI{ALL\_DEBUG} to get extra debugging from all Condor
components.

A daemon that has successfully registered itself with CCB will
advertise this fact in its address in its ClassAd.  
The ClassAd attribute \Attr{MyAddress} will contain information
about its \AdStr{CCBID}.

\subsubsection{Scalability and CCB}

Any number of CCB servers may be used to serve a pool of Condor
daemons.  For example, half of the pool could use one CCB server and
half could use another.  Or for redundancy, all daemons could use both
CCB servers and then CCB connection requests will load-balance
across them.  Typically, the limit of how many daemons may be
registered with a single CCB server depends on the authentication
method used by the \Condor{collector} for DAEMON-level and READ-level access,
and on the amount of memory available to the CCB server.  We are not
able to provide specific recommendations at this time, 
but to give a very rough idea,
a server class machine should be able to handle CCB
service plus normal \Condor{collector} service for a pool containing
a few thousand slots without much trouble.



\input{admin-man/tcp-update.tex}

%%%%%%%%%%%%%%%%%%%%%%%%%%%%%%%%%%%%%%%%%%%%%%%%%%%%%%%%%%%%%%%%%%%%%%%%%%%
\subsection{\label{sec:ipv6}Running Condor on an IPv6 Network Stack}
%%%%%%%%%%%%%%%%%%%%%%%%%%%%%%%%%%%%%%%%%%%%%%%%%%%%%%%%%%%%%%%%%%%%%%%%%%
\index{IPv6|(}

Condor has limited support for running on IPv6 networks.

Current Limitations

\begin{itemize}
    \item{Microsoft Windows platforms are \emph{not} supported.}
    \item{Mixed IPv4/IPv6 pools are \emph{not} supported.}
    \item{Security policies cannot use IP addresses, only host names.
If using \Expr{NO\_DNS=TRUE}, the host names are reformatted IP addresses,
and can be matched against those. }
    \item{\MacroNI{NETWORK\_INTERFACE} \emph{must be set} to a 
specific IPv6 address. 
It is not possible to use multiple IPv6 interfaces on a single computer.}
    \item{There must be valid IPv6 (AAAA) DNS and reverse DNS records for 
the computers. 
Setting the configuration \Expr{NO\_DNS=TRUE} removes this limitation.}
\end{itemize}

Enabling IPv6

\begin{itemize}
    \item{In the configuration, set \Expr{ENABLE\_IPV6 = TRUE}.}
    \item{Specify the IPv6 interface to use. 
Do \emph{not} put square brackets ([]) around this address.
As an example,
\begin{verbatim}
NETWORK_INTERFACE = 2607:f388:1086:0:21b:24ff:fedf:b520
\end{verbatim}
}
\end{itemize}

Additional Notes

Specification of  \MacroNI{CONDOR\_HOST} or \MacroNI{COLLECTOR\_HOST}
as an IP address \emph{must} place the address, 
but \emph{not} the port, in square brackets. 
Host names may be specified. 
For example:

\begin{verbatim}
CONDOR_HOST =[2607:f388:1086:0:21e:68ff:fe0f]:6462
# This configures the collector to listen on the non-standard port 5332.
COLLECTOR_HOST =[2607:f388:1086:0:21e:68ff:fe0f:6462]:5332
\end{verbatim}

Because IPv6 addresses are not currently supported in Condor's 
security settings, 
\MacroUNI{CONDOR\_HOST} or \MacroUNI{COLLECTOR\_HOST} will not
be permitted in the security configuration, 
to specify an IP address.

When using the configuration variable \MacroNI{NO\_DNS},
IPv6 addresses are turned into host names by taking the IPv6 address, 
changing colons to dashes, and appending \MacroUNI{DEFAULT\_DOMAIN\_NAME}. 
So, 
\begin{verbatim}
2607:f388:1086:0:21b:24ff:fedf:b520
\end{verbatim}
becomes 
\begin{verbatim}
2607-f388-1086-0-21b-24ff-fedf-b520.example.com 
\end{verbatim}
assuming
\begin{verbatim}
DEFAULT_DOMAIN_NAME=example.com
\end{verbatim}

\index{IPv6|)}


% NAT -- Network address translation
% when access to internet uses a single IP addr/port,
%  but there are multiple computers communicating by this single
%  place
% The NAT is an extra layer that translates the multiple addr/port
%  to the single, and visa versa (from the single to one of the
%  multiple).

% Lore from Derek:
% however, "nat" is also one of those strange condor-team terms (like
% "frank") that has it's own, special meaning. :)
%
% a "nat" is a unit for measuring productivity (or lack thereof) in
% condor work over time.  it can be normalized to any time unit you
% want, by dividing the amount of work accomplished into the time scale
% you want.  1 nat is *very* little work over a given time period,
% almost too small to measure unless you use a long time scale.  at the
% time it first came up, we decided that jim basney *sleeps* at about 40
% nats, just for comparison. :)


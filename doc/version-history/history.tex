%%%%%%%%%%%%%%%%%%%%%%%%%%%%%%%%%%%%%%%%%%%%%%%%%%%%%%%%%%%%%%%%%%%%%%
\section{\label{sec:History-Intro}Introduction to Condor Versions}
%%%%%%%%%%%%%%%%%%%%%%%%%%%%%%%%%%%%%%%%%%%%%%%%%%%%%%%%%%%%%%%%%%%%%%

This chapter provides descriptions of what features have been added or
bugs fixed for each version of Condor.
The first section describes the Condor version numbering scheme, what
the numbers mean, and what the different \Term{release series} are.
The rest of the sections each describe a specific release series, and
all the Condor versions found in that series.

%%%%%%%%%%%%%%%%%%%%%%%%%%%%%%%%%%%%%%%%%%%%%%%%%%%%%%%%%%%%%%%%%%%%%%
\subsection{\label{sec:Version-Number-Scheme}
Condor Version Number Scheme}
%%%%%%%%%%%%%%%%%%%%%%%%%%%%%%%%%%%%%%%%%%%%%%%%%%%%%%%%%%%%%%%%%%%%%%

Starting with version 6.0.1, Condor adopted a new, hopefully easy to
understand version numbering scheme.
It reflects the fact that Condor is both a production system and a
research project.
The numbering scheme was primarily taken from the Linux kernel's
version numbering, so if you are familiar with that, it should seem
quite natural.

There will usually be two Condor versions available at any given time,
the \Term{stable} version, and the \Term{development} version.
Gone are the days of ``patch level 3'', ``beta2'', or any other random
words in the version string.
All versions of Condor now have exactly three numbers, separated by
``.''   

\begin{itemize}

\item The first number represents the major version number, and will
change very infrequently.

\item \emph{The thing that determines whether a version of Condor is
\Term{stable} or \Term{development} is the second digit.
Even numbers represent stable versions, while odd numbers represent
development versions.}

\item The final digit represents the minor version number, which
defines a particular version in a given release series.

\end{itemize}


%%%%%%%%%%%%%%%%%%%%%%%%%%%%%%%%%%%%%%%%%%%%%%%%%%%%%%%%%%%%%%%%%%%%%%
\subsection{\label{sec:Stable-Series}The Stable Release Series}
%%%%%%%%%%%%%%%%%%%%%%%%%%%%%%%%%%%%%%%%%%%%%%%%%%%%%%%%%%%%%%%%%%%%%%

People expecting the stable, production Condor system should download
the stable version, denoted with an even number in the second digit of
the version string.
Most people are encouraged to use this version.  
We will only offer our paid support for versions of Condor from the
stable release series.

\emph{On the stable series, new minor version releases will only
be made for bug fixes and to support new platforms.}
No new features will be added to the stable series.
People are encouraged to install new stable versions of Condor when
they appear, since they probably fix bugs you care about.
Hopefully, there will not be many minor version releases for any given
stable series.


%%%%%%%%%%%%%%%%%%%%%%%%%%%%%%%%%%%%%%%%%%%%%%%%%%%%%%%%%%%%%%%%%%%%%%
\subsection{\label{sec:Developement-Series}
The Development Release Series}
%%%%%%%%%%%%%%%%%%%%%%%%%%%%%%%%%%%%%%%%%%%%%%%%%%%%%%%%%%%%%%%%%%%%%%

Only people who are interested in the latest research, new features
that haven't been fully tested, etc, should download the development
version, denoted with an odd number in the second digit of the version
string.  
We will make a best effort to ensure that the development series will
work, but we make no guarantees.

On the development series, new minor version releases will probably
happen frequently.
People should not feel compelled to install new minor versions unless
they know they want features or bug fixes from the newer development
version.

\emph{Most sites will probably never want to install a development
version of Condor for any reason.}
Only if you know what you are doing (and like pain), or were
explicitly instructed to do so by someone on the Condor Team, should
you install a development version at your site.

After the feature set of the development series is satisfactory to the
Condor Team, we will put a code freeze in place, and from that point
forward, only bug fixes will be made to that development series.
When we have fully tested this version, we will release a new stable
series, resetting the minor version number, and start work on a new
development release from there.

%%%%%%%%%%%%%%%%%%%%%%%%%%%%%%%%%%%%%%%%%%%%%%%%%%%%%%%%%%%%%%%%%%%%%%
% The rest of this file just inputs other files which contain sections
% describing each release series in detail.
%%%%%%%%%%%%%%%%%%%%%%%%%%%%%%%%%%%%%%%%%%%%%%%%%%%%%%%%%%%%%%%%%%%%%%

% upgrade instructions are in the Pool Management section
%%%%%%%%%%%%%%%%%%%%%%%%%%%%%%%%%%%%%%%%%%%%%%%%%%%%%%%%%%%%%%%%%%%%%%
\section{\label{sec:gotchas}Upgrading from the 7.6 series to the 7.8 series of Condor}
%%%%%%%%%%%%%%%%%%%%%%%%%%%%%%%%%%%%%%%%%%%%%%%%%%%%%%%%%%%%%%%%%%%%%%

\index{upgrading!items to be aware of}
While upgrading from the 7.6 series of Condor to the 7.8 series will bring many
new features and improvements introduced in the 7.7 series of Condor, it will
also introduce changes that administrators of sites running from an older
Condor version should be aware of when planning an upgrade.  Here is a list of
items that administrators should be aware of.

\begin{itemize}

\item In the grid universe, the Amazon grid-type is gone and has been replaced
	with the EC2 grid-type.  Also, support for grid-type gt4 (Web Services
	GRAM) has been removed.

\item Default job submit options related to file transfers have changed. 
Across all platforms, defaults are now
\begin{verbatim}
  should_transfer_files = IF_NEEDED
  when_to_transfer_output = ON_EXIT
\end{verbatim}
See section~\ref{sec:file-transfer-if-when} for details.

\item  On Linux and Mac OS, common utility code is now contained in a set of
shared libraries. In the Linux native packages, most of these libraries
are placed under \File{/usr/lib[64]/condor} and the RUNPATH attribute is set in
the binaries to search there for them.
In the tarball packages, these libraries are placed under \File{lib} and
\File{lib/condor}, and the RUNPATH attribute is set in the binaries to search
for them under the relative paths \File{../lib} and \File{../lib/condor}.
This means that if you move or copy a Condor binary from a tarball
package to a different location, you must do one of the following:
\begin{itemize}
	\item Move or copy the corresponding \File{lib/} directory with it, or
  \item Make a symlink in the new location pointing back to the original \File{lib/}
  directory, or
  \item Set environment variable \Env{LD\_LIBRARY\_PATH} to point to the original \File{lib/} and \File{lib/condor/}
  directories
\end{itemize}
One of the new shared libraries, \File{libcondor\_utils\_7\_8\_0}, has no \File{.so}
versioning. Instead, the Condor version is included in the library name.
This means that a Condor binary must always be matched with the
\File{libcondor\_utils} library from the same Condor release.


\item  The \Condor{hdfs} service is no longer included within the Condor
	release.  Instead, the Condor + HDFS integration previously bundled with
	version 7.6 is available in version 7.8 as a \Term{Contribution Module}.
	Contribution Modules are optional packages that add functionality to
	Condor, but are provided and maintained outside of the core code base.  See
	the Condor Wiki at
	\URL{https://condor-wiki.cs.wisc.edu/index.cgi/wiki?p=ContribModules}.

\item Previous to version 7.8, by default the \Condor{master} would restart any
	individual daemon under its control if it notices that the file
	modification time of the binary for that daemon has changed.  Now the
	\Condor{master} will only monitor the file modification time of the
	\Condor{master} binary itself.  See section~\ref{sec:Pool-Upgrade}.  Also,
	see \MacroNI{MASTER\_NEW\_BINARY\_RESTART} on
	page~\pageref{param:MasterNewBinaryRestart}.

\item In DAGMan, if you have a PRE and a POST script on a node, the default now
	is that the POST script is run even if the PRE script failed.   This change
	could impact unaware workflows such that POST scripts might erroneously
	report the node as succeeded. You can get the old behavior by setting
	\MacroNI{DAGMAN\_ALWAYS\_RUN\_POST} to False.  In addition, you can no
	longer directly submit a rescue DAG file with \Condor{submit\_dag} unless
	\MacroNI{DAGMAN\_WRITE\_PARTIAL\_RESCUE} is set to False (not normally
	recommended).  See section~\ref{sec:DAGMan}.

\item The \MacroNI{KILL} expression cannot be used to grant more time to a job
	than offered by \Macro{MachineMaxVacateTime}. In Condor v7.8 and above, it
	is anticipated that most sites will simply use a default value of False for
	\MacroNI{KILL} and set \MacroNI{MachineMaxVacateTime} to control how long
	to wait.  See page~\pageref{param:MachineMaxVacateTime} for more
	information.


\end{itemize}


%%%      PLEASE RUN A SPELL CHECKER BEFORE COMMITTING YOUR CHANGES!
%%%      PLEASE RUN A SPELL CHECKER BEFORE COMMITTING YOUR CHANGES!
%%%      PLEASE RUN A SPELL CHECKER BEFORE COMMITTING YOUR CHANGES!
%%%      PLEASE RUN A SPELL CHECKER BEFORE COMMITTING YOUR CHANGES!
%%%      PLEASE RUN A SPELL CHECKER BEFORE COMMITTING YOUR CHANGES!

%%%%%%%%%%%%%%%%%%%%%%%%%%%%%%%%%%%%%%%%%%%%%%%%%%%%%%%%%%%%%%%%%%%%%%
\section{\label{sec:History-7-9}Development Release Series 7.9}
%%%%%%%%%%%%%%%%%%%%%%%%%%%%%%%%%%%%%%%%%%%%%%%%%%%%%%%%%%%%%%%%%%%%%%

This is the development release series of Condor.
The details of each version are described below.

%%%%%%%%%%%%%%%%%%%%%%%%%%%%%%%%%%%%%%%%%%%%%%%%%%%%%%%%%%%%%%%%%%%%%%
\subsection*{\label{sec:New-7-9-1}Version 7.9.1}
%%%%%%%%%%%%%%%%%%%%%%%%%%%%%%%%%%%%%%%%%%%%%%%%%%%%%%%%%%%%%%%%%%%%%%

\noindent Release Notes:

\begin{itemize}

\item Condor version 7.9.1 not yet released.
%\item Condor version 7.9.1 released on Month Date, 2012.

\item Condor no longer looks for its main configuration file in the
location \File{\MacroUNI{GLOBUS\_LOCATION}/etc/condor\_config}.
\Ticket{2830}

\end{itemize}


\noindent New Features:

\begin{itemize}

\item \Condor{job\_router} can now submit the routed copy of jobs to a
different \Condor{schedd} than the one that serves as the source of
jobs to be routed.  The spool directories of the two
\Condor{schedds} must still be directly accessible to
\Condor{job\_router}.  This feature is enabled by using the new
optional configuration settings:

\begin{itemize}
\item \Macro{JOB\_ROUTER\_SCHEDD1\_SPOOL}
\item \Macro{JOB\_ROUTER\_SCHEDD2\_SPOOL}
\item \Macro{JOB\_ROUTER\_SCHEDD1\_NAME}
\item \Macro{JOB\_ROUTER\_SCHEDD2\_NAME}
\item \Macro{JOB\_ROUTER\_SCHEDD1\_POOL}
\item \Macro{JOB\_ROUTER\_SCHEDD2\_POOL}
\end{itemize}
\Ticket{3030}

\item \Condor{defrag} now has a policy option to cancel draining
of a machine that is in draining mode.  This can be used to effect
partial draining of machines.
\Ticket{2993}

\end{itemize}

\noindent Configuration Variable and ClassAd Attribute Additions and Changes:

\begin{itemize}

\item Added the attribute \Attr{DAGManNodesMask} to control the verboseness of
the log referred to by \Attr{DAGManNodesLog}.

\item The new configuration variable
\Macro{QUEUE\_SUPER\_USER\_MAY\_IMPERSONATE} specifies a regular
expression that matches the user names that
the queue super user may impersonate when managing jobs.  When not
set, the default behavior is to allow impersonation of any user who
has had a job in the queue during the life of the \Condor{schedd}.  For
proper functioning of the \Condor{shadow}, the \Condor{gridmanager}, and
the \Condor{job\_router}, this expression, if set, must match the owner
names of all jobs that these daemons will manage.
\Ticket{3030}

\label{JobRouterSchedd1Spool}
\item[\Macro{JOB\_ROUTER\_SCHEDD1\_SPOOL}]
  The path to the spool directory for the \Condor{schedd} serving as the
  source of jobs for routing.  If not specified, this defaults to
  \Macro{SPOOL}.  If specified, this parameter must point to the spool
  directory of the \Condor{schedd} identified by \Macro{JOB\_ROUTER\_SCHEDD1\_NAME}.

\label{JobRouterSchedd2Spool}
\item[\Macro{JOB\_ROUTER\_SCHEDD2\_SPOOL}]
  The path to the spool directory for the \Condor{schedd} to which the
  routed copy of the jobs are submitted.  If not specified, this defaults to
  \Macro{SPOOL}.  If specified, this parameter must point to the spool
  directory of the \Condor{schedd} identified by \Macro{JOB\_ROUTER\_SCHEDD2\_NAME}.

\label{JobRouterSchedd1Name}
\item[\Macro{JOB\_ROUTER\_SCHEDD1\_NAME}]
  The advertised daemon name of the \Condor{schedd} serving as the
  source of jobs for routing.  If not specified, this defaults to the
  local \Condor{schedd}.  If specified, this parameter must name the
  same \Condor{schedd} whose spool is configured in
  \Macro{JOB\_ROUTER\_SCHEDD1\_SPOOL}.  If the named \Condor{schedd} is
  not advertised in the local pool, \Macro{JOB\_ROUTER\_SCHEDD1\_POOL}
  will also need to be set.

\label{JobRouterSchedd2Name}
\item[\Macro{JOB\_ROUTER\_SCHEDD2\_NAME}]
  The advertised daemon name of the \Condor{schedd} to which the
  routed copy of the jobs are submitted.  If not specified, this defaults to
  the local \Condor{schedd}.  If specified, this parameter must name the
  same \Condor{schedd} whose spool is configured in
  \Macro{JOB\_ROUTER\_SCHEDD2\_SPOOL}.  If the named \Condor{schedd} is
  not advertised in the local pool, \Macro{JOB\_ROUTER\_SCHEDD2\_POOL}
  will also need to be set.

\label{JobRouterSchedd1Pool}
\item[\Macro{JOB\_ROUTER\_SCHEDD1\_POOL}]
  The Condor pool (collector address) of the \Condor{schedd} serving as the
  source of jobs for routing.  If not specified, this defaults to the
  local pool.

\label{JobRouterSchedd2Pool}
\item[\Macro{JOB\_ROUTER\_SCHEDD2\_POOL}]
  The Condor pool (collector address) of the \Condor{schedd} to which
  the routed copy of the jobs are submitted.  If not specified, this
  defaults to the local pool.

\label{JobRouterSchedd2Pool}
\item[\Macro{DEFRAG\_WHOLE\_MACHINE\_EXPR}]
  This new configuration value is an expression that specifies
  which draining machines should have draining be canceled.  This defaults
  to \Macro{DEFRAG\_WHOLE\_MACHINE\_EXPR}.  This could be used to drain
  partial rather than whole machines.
\Ticket{2993}

\end{itemize}

\noindent Bugs Fixed:

\begin{itemize}

\item The ClassAd functions \Procedure{splitUserName} and \Procedure{splitSlotName}
leaked a small amount of memory every time they were evaluated.  This bug was
introduced when these functions were added in 7.7.6.
\Ticket{3082}

\end{itemize}

\noindent Known Bugs:

\begin{itemize}

\item None.

\end{itemize}

\noindent Additions and Changes to the Manual:

\begin{itemize}

\item None.

\end{itemize}


%%%%%%%%%%%%%%%%%%%%%%%%%%%%%%%%%%%%%%%%%%%%%%%%%%%%%%%%%%%%%%%%%%%%%%
\subsection*{\label{sec:New-7-9-0}Version 7.9.0}
%%%%%%%%%%%%%%%%%%%%%%%%%%%%%%%%%%%%%%%%%%%%%%%%%%%%%%%%%%%%%%%%%%%%%%

\noindent Release Notes:

\begin{itemize}

\item Condor version 7.9.0 not yet released.
%\item Condor version 7.9.0 released on Month Date, 2012.

\end{itemize}


\noindent New Features:

\begin{itemize}

\item Machine slots can now be configured to identify and
divide customized local resources.
Jobs may then request these resources.
See section~\ref{sec:Configuring-SMP} for details.
\Ticket{2905}

\item Condor now supports and implements the caching of ClassAds 
to reduce memory footprints. Caching may be disabled by setting
the new configuration variable \Macro{ENABLE\_CLASSAD\_CACHING}
to \Expr{False}.
Third party use of the ClassAd library does not enable the
caching of ClassAds by default; 
Condor does enable the caching of ClassAds by default.
\Ticket{2541}
\Ticket{3127}

\item \Condor{status} now returns the \Condor{schedd} ClassAd directly 
from the \Condor{schedd} daemon,
if both options \Opt{-direct} and \Opt{-schedd} are given on the command line.
\Ticket{2492}

\item The new \Opt{-status} and \Opt{-echo} command line options to 
\Condor{wait} command cause it to show job start and terminate information,
and to print events to \Code{stdout}.
\Ticket{2926}

\item Added a \Expr{DEBUG} logging level output flag \Dflag{CATEGORY},
which causes Condor to include the logging level
flags in effect for each line of logged output.
\Ticket{2712}

\item \Condor{status} and \Condor{q} each have a new \Opt{-autoformat} option
to make some output format specifications easier than the existing
\Opt{-format} option.
See the \Condor{status} manual page located on page~\pageref{man-condor-status}
and the \Condor{q} manual page located on page~\pageref{man-condor-q} 
for details.
\Ticket{2941}

\item Enhanced the ClassAd log system to report the log line number 
on parse failures, 
and improved the ability to detect parse failures closer to 
the point of corruption.
\Ticket{2934}

\item Added an \Opt{-evaluate} option to \Condor{config\_val}, which causes the configured value queried from
a given daemon to be evaluated with respect to that daemon's ClassAd.
\Ticket{856}

\item Added code to \Condor{dagman},
such that a \Expr{VARS} assignment in a top-level DAG is applied to splices.
\Ticket{1780}

\item Condor now uses libraries from Globus 5.2.1.
\Ticket{2838}

\item When authenticating Condor daemons with GSI and
configuration variable \MacroNI{GSI\_DAEMON\_NAME} is undefined, 
Condor checks that the server name in the certificate matches the 
host name that the client is connecting to. 
When \MacroNI{GSI\_DAEMON\_NAME} is defined,
the old behavior is preserved: only certificates matching
\MacroNI{GSI\_DAEMON\_NAME} pass the authentication step, 
and no host name check is performed.  
The behavior of the host name check
may be further controlled with the new configuration variables
\MacroNI{GSI\_SKIP\_HOST\_CHECK} and
\MacroNI{GSI\_SKIP\_HOST\_CHECK\_CERT\_REGEX}.
\Ticket{1605}

\item Added new capability to \Condor{submit} to allow recursive macros in
submit description files. 
This facility allows one to update variables recursively. 
Before this new capability was added,
recursive definition would send \Condor{submit} into an
infinite loop of expanding the macro,
such that the expansion would fill up memory.
See section~\ref{macro-in-submit-description-file} for details.
\Ticket{406}

\item A DAGMan limitation and restriction has been removed.  
It is now permitted to define a \SubmitCmd{log} command using a macro,
within a node job's submit description file.
\Ticket{2428}

\item To enforce the dependencies of a DAG,
DAGMan now uses and watches only the default node
user log of the \Condor{dagman} job for events.  
DAGMan requests the \Condor{schedd} and \Condor{shadow} daemons to write each
event to this default log, 
in addition to writing to a log specified by the node job.
\Condor{dagman} writes POST script terminate events only to its default log;
these terminate events are not written to the user log.
This behavior can be turned off by setting the configuration variable
\Macro{DAGMAN\_ALWAYS\_USE\_NODE\_LOG} to \Expr{False}.
For correct behavior,
\MacroNI{DAGMAN\_ALWAYS\_USE\_NODE\_LOG} should be set to \Expr{False}
if \Condor{dagman} version 7.9.0 or later is submitting jobs 
to an older version of
a \Condor{schedd} daemon or of a \Condor{submit} executable.
\Ticket{2807}

\end{itemize}

\noindent Configuration Variable and ClassAd Attribute Additions and Changes:

\begin{itemize}

\item The new configuration variables \Macro{MACHINE\_RESOURCE\_NAMES}
(see section~\ref{param:MachineResourceNames})
and \Macro{MACHINE\_RESOURCE\_<name>}
(see section~\ref{param:MachineResourceResourcename})
identify and specify the use of customized local machine resources.
\Ticket{2905}

\item The new configuration variable \MacroNI{ENABLE\_CLASSAD\_CACHING}
controls whether the new caching feature of ClassAds is used.
The default value is \Expr{True}.
\Ticket{3127}

\item The new configuration variable \Macro{CLASSAD\_LOG\_STRICT\_PARSING}
controls whether ClassAd log files such as the job queue
log are read with strict parse checking on ClassAd expressions.
\Ticket{3069}

\item The default value for configuration variable \Macro{USE\_PROCD}
is now \Expr{True} for the \Condor{master} daemon.  
This means that by
default the \Condor{master} will start a \Condor{procd} daemon to be used 
by it and all other daemons on that machine.
\Ticket{2911}

\item There is a new configuration variable used by the \Condor{starter}.
If \Macro{STARTER\_RLIMIT\_AS} is set to an integer value, 
the \Condor{starter}
will use the \Procedure{setrlimit} system call with the 
\Code{RLIMIT\_AS} parameter to
limit the virtual memory size of each process in the user job.  
The value of this configuration variable is a ClassAd expression, 
evaluated in the context of both the machine and job ClassAds, 
where the machine ClassAd is the \Expr{my} ClassAd, 
and the job ClassAd is the \Expr{target} ClassAd.
\Ticket{1663}

\item New configuration variables were added to to the \Condor{schedd} to
define statistics that count subsets of jobs. 
These variables have the form \Macro{SCHEDD\_COLLECT\_STATS\_BY\_<Name>},
and should be defined by a ClassAd expression that evaluates to a string.
See section~\ref{param:ScheddCollectStatsByName}
for the complete definition.
The optional configuration variable of the form
\Macro{SCHEDD\_EXPIRE\_STATS\_BY\_<Name>} can be used to set an expiration time,
in seconds, for each set of statistics.
\Ticket{2862}

\item The new \SubmitCmd{batch\_queue} submit description file command
and new job ClassAd attribute \Attr{BatchQueue} specify which job
queue to use for grid universe jobs of type
\SubmitCmd{pbs}, \SubmitCmd{lsf}, and \SubmitCmd{sge}.
\Ticket{2996}

\item The new configuration variable \Macro{GSI\_SKIP\_HOST\_CHECK} is
a boolean that controls whether a check is performed during
GSI authentication of a Condor daemon.  
When the default value \Expr{False},
the check is not skipped, so the daemon host name must match the
host name in the daemon's certificate, unless otherwise exempted
by values of \MacroNI{GSI\_DAEMON\_NAME} or
\MacroNI{GSI\_SKIP\_HOST\_CHECK\_CERT\_REGEX}.
When \Expr{True}, this check is skipped, and hosts will not be rejected
due to a mismatch of certificate and host name.
\Ticket{1605}

\item The new configuration variable
\MacroNI{GSI\_SKIP\_HOST\_CHECK\_CERT\_REGEX} may be set to a
regular expression.  GSI certificates of Condor daemons with a
subject name that are matched in full by this regular expression
are not required to have a matching daemon host name and certificate
host name.  The default is an empty regular expression, which will
not match any certificates, even if they have an empty subject name.
\Ticket{1605}

\end{itemize}

\noindent Bugs Fixed:

\begin{itemize}

\item Fixed a bug in which usage of cgroups incorrectly included the page cache 
in the maximum memory usage.
This bug fix is also included in Condor version 7.8.2.
\Ticket{3003}

\item The EC2 GAHP will now respect the value of the environment variable
\Env{X509\_CERT\_DIR} and the configuration variable
\Macro{GSI\_DAEMON\_TRUSTED\_CA\_DIR} for \emph{all} secure connections.
\Ticket{2823}

\item Condor will avoid selecting down (disabled) network interfaces.  Previously Condor could select a down interface over an up (active) interface.
\Ticket{2893}

\item Made logic in the \Condor{negotiator} that computes submitter limits 
properly aware of the configuration variable
\Macro{NEGOTIATOR\_CONSIDER\_PREEMPTION}.
\Ticket{2952}


\item Condor no longer back-dates file modification times by 3 minutes
when transferring job input files into the job spool directory or the job
execute directory.
\Ticket{2423}

\item Fixed a bug in which the use of a pipe in the configuration file 
on Windows platforms would cause a visible console window 
to show up whenever the configuration was read.
\Ticket{1534}

\end{itemize}

\noindent Known Bugs:

\begin{itemize}

\item None.

\end{itemize}

\noindent Additions and Changes to the Manual:

\begin{itemize}

\item Machine ClassAd attribute string values relating to \Attr{OpSys} have
been documented for Scientific Linux platforms.
\Ticket{2366}

\end{itemize}



%%%      PLEASE RUN A SPELL CHECKER BEFORE COMMITTING YOUR CHANGES!
%%%      PLEASE RUN A SPELL CHECKER BEFORE COMMITTING YOUR CHANGES!
%%%      PLEASE RUN A SPELL CHECKER BEFORE COMMITTING YOUR CHANGES!
%%%      PLEASE RUN A SPELL CHECKER BEFORE COMMITTING YOUR CHANGES!
%%%      PLEASE RUN A SPELL CHECKER BEFORE COMMITTING YOUR CHANGES!

%%%%%%%%%%%%%%%%%%%%%%%%%%%%%%%%%%%%%%%%%%%%%%%%%%%%%%%%%%%%%%%%%%%%%%
\section{\label{sec:History-7-8}Stable Release Series 7.8}
%%%%%%%%%%%%%%%%%%%%%%%%%%%%%%%%%%%%%%%%%%%%%%%%%%%%%%%%%%%%%%%%%%%%%%

This is a stable release series of Condor.
As usual, only bug fixes (and potentially, ports to new platforms)
will be provided in future 7.8.x releases.
New features will be added in the 7.9.x development series.

The details of each version are described below.

%%%%%%%%%%%%%%%%%%%%%%%%%%%%%%%%%%%%%%%%%%%%%%%%%%%%%%%%%%%%%%%%%%%%%%
\subsection*{\label{sec:New-7-8-2}Version 7.8.2}
%%%%%%%%%%%%%%%%%%%%%%%%%%%%%%%%%%%%%%%%%%%%%%%%%%%%%%%%%%%%%%%%%%%%%%

\noindent Release Notes:

\begin{itemize}

\item Condor version 7.8.2 not yet released.
%\item Condor version 7.8.2 released on Month Date, 2012.

\end{itemize}


\noindent New Features:

\begin{itemize}

\item The \File{libcondorapi} library for reading and writing job event
logs is again available as a shared library on Linux and Mac OS platforms.
Since Condor 7.5.x, it had only been available as a static library.
\Ticket{3047}

\end{itemize}

\noindent Configuration Variable and ClassAd Attribute Additions and Changes:

\begin{itemize}

\item To avoid the output of an unnecessary DAGMan error message,
the value of \Macro{DAGMAN\_LOG\_ON\_NFS\_IS\_ERROR}
is ignored when both \MacroNI{CREATE\_LOCKS\_ON\_LOCAL\_DISK}
and \MacroNI{ENABLE\_USERLOG\_LOCKING} are \Expr{True}.
\Ticket{3087}

\end{itemize}

\noindent Bugs Fixed:

\begin{itemize}

\item Fixed a bug in which usage of cgroups incorrectly included the
page cache in the maximum memory usage.
This bug fix is also included in Condor version 7.9.0.
\Ticket{3003}

\item Jobs from a hook to fetch work, 
where the hook is defined by configuration variable 
\MacroNI{<Keyword>\_HOOK\_FETCH\_WORK},
now correctly receive dynamic slots from a partitionable slot 
instead of claiming the entire partitionable slot.
\Ticket{2819}

\item Fixed a bug in which a slot might become stuck in the Preempting state
when a \Condor{startd} is configured with a hook to fetch work,
as defined by \Macro{<Keyword>\_HOOK\_FETCH\_WORK}.
\Ticket{3076}

\item Fixed a bug that caused Condor to transfer a job's input files from
the execute machine back to the submit machine as if they were output files.
This would happen if the
job's input files were stored in Condor's spool directory;
occurred if the job was submitted via Condor-C or via 
\Condor{submit} with the \Opt{-spool} or \Opt{-remote} options.
\Ticket{2406}

\item Fixed a bug that could cause the first grid-type cream jobs destined 
for a particular CREAM server to never be submitted to that server.
This bug was probably introduced in Condor version 7.6.5.
\Ticket{3054}

\item Fixed several problems with the XML parsing class
\Code{ClassAdXMLParser} in the ClassAds library:
  \begin{itemize}
  \item Several methods named \Procedure{ParseClassAd} were declared, 
  but never implemented. 
\Ticket{3049}
  \item The parser silently dropped leading white space in string values.
\Ticket{3042}
  \item The parser could go into an infinite loop or leak memory when
    reading a malformed ClassAd XML document. 
\Ticket{3045}
  \end{itemize}

\item Fixed a bug that prevented the \Opt{-f} command line option to
\Condor{history} from being recognized.
The \Opt{-f} option was being interpreted as \Opt{-forward}. 
At least four letters are now required for the \Opt{-forward} option
(\Opt{-forw}) to prevent ambiguity.
\Ticket{3044}

\item The implementation of the \Condor{history} \Opt{-backwards} option, 
which is the default ordering for reading the history file,
in the 7.7 series did not work on Windows platforms.
This has been fixed.
\Ticket{3055}

\item Fixed a bug that caused an invalid proxy to be delegated when
refreshing the job's X.509 proxy when configuration variable
\Macro{DELEGATE\_JOB\_GSI\_CREDENTIALS\_LIFETIME} was set to 0.
\Ticket{3059}

\item Fixed a bug in which DAGMan did not account properly for jobs being
suspended and then unsuspended.
\Ticket{3108}

\item Job IDs generated by NorduGrid ARC 12.05 and above are now
properly recognized.
\Ticket{3062}

\item Fixed a bug in which Condor would not mark grid-type nordugrid jobs
as Running due to variation in the format of the job status value.
NorduGrid ARC job statuses of the form \Expr{INLRMS: ?} are now
properly recognized both with and without the space after the colon.
\Ticket{3118}

\item The \Condor{gridmanager} now properly handles X.509 proxy files
that are specified in the job ClassAd with a relative path name.
\Ticket{3027}

\item Fixed a bug that caused daemon names,
as set in configuration variables such as \MacroNI{STARTD\_NAME},
containing a period character to be ignored.
\Ticket{3172}

\item Fixed a bug that prevented the \Condor{schedd} from removing old
execute directories for local universe jobs on startup.
\Ticket{3176}

\item \Condor{dagman} now takes note of \Macro{ULOG\_JOB\_RECONNECT\_FAILED} 
events in the user log for counting idle jobs.
\Ticket{3189}

\end{itemize}

\noindent Known Bugs:

\begin{itemize}

\item None.

\end{itemize}

\noindent Additions and Changes to the Manual:

\begin{itemize}

\item None.

\end{itemize}


%%%%%%%%%%%%%%%%%%%%%%%%%%%%%%%%%%%%%%%%%%%%%%%%%%%%%%%%%%%%%%%%%%%%%%
\subsection*{\label{sec:New-7-8-1}Version 7.8.1}
%%%%%%%%%%%%%%%%%%%%%%%%%%%%%%%%%%%%%%%%%%%%%%%%%%%%%%%%%%%%%%%%%%%%%%

\noindent Release Notes:

\begin{itemize}

\item Condor version 7.8.1 released on June 15, 2012.

\end{itemize}


\noindent New Features:

\begin{itemize}

\item None.

\end{itemize}

\noindent Configuration Variable and ClassAd Attribute Additions and Changes:

\begin{itemize}

\item (Added in 7.8.0.) The new configuration variable
\Macro{ENABLE\_DEPRECATION\_WARNINGS} causes \Condor{submit} to issue
warnings when a job requests features that are no longer supported.
\Ticket{2968}

\item (Added in 7.7.6) The new configuration variable
\Macro{BATCH\_GAHP} should be used instead of \Macro{PBS\_GAHP},
\Macro{LSF\_GAHP} and \Macro{SGE\_GAHP}. These older configuration
variables are still recognized, but their use is now discouraged.
\Ticket{2670}

\item The default value for \Macro{GROUP\_SORT\_EXPR} was changed 
so that the \Expr{<none>} group would always negotiate last 
when using hierarchical group quotas.
Associated with that, 
the default value for \Macro{NEGOTIATOR\_ALLOW\_QUOTA\_OVERSUBSCRIPTION} 
was changed to \Expr{True}.  
These changes were made to make negotiation behave more like it did 
in the stable 7.4 series of Condor,
before hierarchical group quotas were added.
\Ticket{3040}

\end{itemize}

\noindent Bugs Fixed:

\begin{itemize}

\item Fixed a bug that caused events to not be written to the job event
log when the log is written in XML and a job policy expression triggering
the event contains any double quote marks.
\Ticket{3048}

\item Fixed a bug in the Condor init script that would cause
the init script to hang if Condor was not running.
\Ticket{2872}

\item Fixed a bug that caused parallel universe jobs using
Parallel Scheduling Groups 
(see section ~\ref{sec:Configure-Dedicated-Groups})
to occasionally stay idle even when
there were available machines to run them.
\Ticket{3017}

\item Fixed a bug that caused the \Condor{gridmanager} to crash when
attempting to submit jobs to a local PBS, LSF, or SGI cluster.
\Ticket{3014}

\item Fixed a bug in the handling of local universe jobs which caused
the \Condor{schedd} to log a spurious \Expr{ERROR} message
every time a local universe job exited, 
and then further caused the statistics for local universe jobs to be 
incorrectly computed.
\Ticket{3008}

\item Changed the internally used \Condor{ckpt\_probe} executable
to link statically, which should make the
checkpoint signature more resistant to non-significant changes in the system
configuration.
\Ticket{2901}

\item Restored Globus and VOMS support for the Mac OS X platform.
\Ticket{2991}

\item Fixed a bug when Condor runs under the PrivSep model,
in which if a job created a hard link from one file to another,
Condor was unable to transfer the files back to the submit side,
and the job was put on hold.
\Ticket{2987}

\item When configuration variables \MacroNI{MaxJobRetirementTime} or
\MacroNI{MachineMaxVacateTime} were very large, estimates of machine
draining badput and completion time were sometimes nonsensical
because of integer overflow.
\Ticket{3001}

\item Fixed a bug where per-job subdirectories and their contents
in \MacroUNI{SPOOL} would not be removed when the associated job
left the queue.
\Ticket{2942}

\item Fixed a bug that could cause the \Condor{schedd} to 
occasionally crash due to a race condition when running local universe jobs.
Associated with the bug would be the error message
\footnotesize
\begin{verbatim}
No local universe jobs were expected to be running, but one just exited!
\end{verbatim}
\normalsize
\Ticket{3009}

\end{itemize}

\noindent Known Bugs:

\begin{itemize}

\item None.

\end{itemize}

\noindent Additions and Changes to the Manual:

\begin{itemize}

\item Submit description file commands introduced in Condor version 7.7.1
have now been documented.
See the \Condor{submit} manual page at ~\ref{man-condor-submit} for
the newly added definitions of
\begin{description}
  \item[\SubmitCmd{ec2\_availability\_zone}]
  \item[\SubmitCmd{ec2\_ebs\_volumes}]
  \item[\SubmitCmd{ec2\_elastic\_ip}]
  \item[\SubmitCmd{ec2\_keypair\_file}]
  \item[\SubmitCmd{ec2\_vpc\_ip}]
  \item[\SubmitCmd{ec2\_vpc\_subnet}]
\end{description}

\item There is now a manual page for \Condor{router\_rm}, 
a script that provides additional features convenient for removing
jobs managed by the Condor Job Router.

\item Documentation not completed for the 7.7.6 release is now available.
The use of configuration variable \MacroNI{BATCH\_GAHP},
as well as the use of the new \SubmitCmd{grid\_resource} of
type \Expr{batch} for local submission of PBS, LSF, and SGE
jobs is documented.
See section ~\ref{sec:batch} for details.
\Ticket{2670}

\end{itemize}


%%%%%%%%%%%%%%%%%%%%%%%%%%%%%%%%%%%%%%%%%%%%%%%%%%%%%%%%%%%%%%%%%%%%%%
\subsection*{\label{sec:New-7-8-0}Version 7.8.0}
%%%%%%%%%%%%%%%%%%%%%%%%%%%%%%%%%%%%%%%%%%%%%%%%%%%%%%%%%%%%%%%%%%%%%%

\noindent Release Notes:

\begin{itemize}

\item Condor version 7.8.0 released on May 10, 2012.

\end{itemize}


\noindent New Features:

\begin{itemize}

\item (Added in 7.7.6.)  The new \Arg{-\_condor\_relocatable} argument
may be given as part of the invocation of a program that uses
standalone checkpointing.  This allows checkpointed programs to restart
without attempting to change to their original directory.
\Ticket{2877}

\item (Added in 7.7.5.) Added the \Arg{-absent} flag to \Condor{status},
which displays absent ClassAds.
\Ticket{2690}

\item (Added in 7.7.5.) Implement absent ads, which help track pool membership
in a persistent way.
\Ticket{2608} 

\end{itemize}

\noindent Configuration Variable and ClassAd Attribute Additions and Changes:

\begin{itemize}

\item The job ClassAd attribute \Attr{RemotePool} is now saved in
  \Attr{LastRemotePool} when the job finishes running.

\end{itemize}

\noindent Bugs Fixed:

\begin{itemize}

\item (Fixed in 7.7.6.) Fix \Arg{-absent}, \Arg{-vm}, and \Arg{-java}
flags to \Condor{status} so that they work with the \Arg{-long} option.
\Ticket{2943}

\item Support glob() on Scientific Linux 6 and others using the new
Linux system call fstatat(), but only when not using remote system calls.
\Ticket{2945}

\item Fixed potential startd crash introduced in v7.7.5 when claiming 
a partitionable slot that was in the Owner state. 
\Ticket{2936}

\item When ClassAd function stringListMember() is called with an empty
string as the second argument, it now evaluates to \Expr{False}.
Previously, it incorrectly evaluated to \Expr{Undefined}.
\Ticket{2953}

\item Format tags \%v and \%V for the \Opt{-format} option now properly
print all ClassAd value types. Previously, \Expr{True} and \Expr{False}
were printed as integers, and new ClassAd types like lists and nested
ClassAds could not be printed.
\Ticket{2960}

\item Restored RCS keyword strings CondorVersion and CondorPlatform to
the Condor binaries. These strings are found and printed by the 
\Opt{ident} program on Unix. They were missing in Condor versions 7.7.3
and later.
\Ticket{2932}

\item \Condor{job\_router} failed to route spooled source jobs.
\Ticket{2955}

\item Fixed a bug on Debian 6 and RHEL 6 that could cause standard
universe jobs to never checkpoint. This would happen if the job
triggered a call to NSCD (Name Service Caching Daemon) but NSCD 
wasn't running. 
Calls to NSCD can be triggered by a look up of a user account or
resolving a machine hostname to an IP address.
Now, NSCD is never consulted by a standard universe
job (this was already the behavior on other platforms).
\Ticket{2973}

\end{itemize}

\noindent Known Bugs:

\begin{itemize}

\item None.

\end{itemize}

\noindent Additions and Changes to the Manual:

\begin{itemize}

\item None.

\end{itemize}



%%%      PLEASE RUN A SPELL CHECKER BEFORE COMMITTING YOUR CHANGES!
%%%      PLEASE RUN A SPELL CHECKER BEFORE COMMITTING YOUR CHANGES!
%%%      PLEASE RUN A SPELL CHECKER BEFORE COMMITTING YOUR CHANGES!
%%%      PLEASE RUN A SPELL CHECKER BEFORE COMMITTING YOUR CHANGES!
%%%      PLEASE RUN A SPELL CHECKER BEFORE COMMITTING YOUR CHANGES!

%%%%%%%%%%%%%%%%%%%%%%%%%%%%%%%%%%%%%%%%%%%%%%%%%%%%%%%%%%%%%%%%%%%%%%
\section{\label{sec:History-7-7}Development Release Series 7.7}
%%%%%%%%%%%%%%%%%%%%%%%%%%%%%%%%%%%%%%%%%%%%%%%%%%%%%%%%%%%%%%%%%%%%%%

This is the development release series of Condor.
The details of each version are described below.


%%%%%%%%%%%%%%%%%%%%%%%%%%%%%%%%%%%%%%%%%%%%%%%%%%%%%%%%%%%%%%%%%%%%%%
\subsection*{\label{sec:New-7-7-5}Version 7.7.5}
%%%%%%%%%%%%%%%%%%%%%%%%%%%%%%%%%%%%%%%%%%%%%%%%%%%%%%%%%%%%%%%%%%%%%%

\noindent Release Notes:

\begin{itemize}

\item Condor version 7.7.5 not yet released.
%\item Condor version 7.7.5 released on Month Date, 2011.

\end{itemize}


\noindent New Features:

\begin{itemize}

\item  \Condor{q} \Opt{-run} now displays the value of the job ClassAd
attribute \Attr{EC2RemoteVirtualMachineName} instead of
\Expr{[????????????????]},
under the HOST(S) column for grid type ec2 jobs.
\Ticket{2599}

\end{itemize}

\noindent Configuration Variable and ClassAd Attribute Additions and Changes:

\begin{itemize}

\item The new configuration variable \Macro{JOB\_QUEUE\_LOG} 
specifies an alternative path and file name for the \File{job\_queue.log} file.
The default value is \File{\MacroUNI{SPOOL}/job\_queue.log}.
This alternative location can be
useful if there is a solid state drive which is big enough to hold the
frequently written to \File{job\_queue.log},
but not big enough to hold the whole contents of the spool directory.
\Ticket{2598}

\end{itemize}

\noindent Bugs Fixed:

\begin{itemize}

\item Communication errors were not always correctly handled when
fetching results of a query when using the \Opt{-stream} option to
\Condor{q}.  This problem was introduced in 7.7.0.
\Ticket{2601}

\end{itemize}

\noindent Known Bugs:

\begin{itemize}

\item None.

\end{itemize}

\noindent Additions and Changes to the Manual:

\begin{itemize}

\item None.

\end{itemize}


%%%%%%%%%%%%%%%%%%%%%%%%%%%%%%%%%%%%%%%%%%%%%%%%%%%%%%%%%%%%%%%%%%%%%%
\subsection*{\label{sec:New-7-7-4}Version 7.7.4}
%%%%%%%%%%%%%%%%%%%%%%%%%%%%%%%%%%%%%%%%%%%%%%%%%%%%%%%%%%%%%%%%%%%%%%

\noindent Release Notes:

\begin{itemize}

\item Condor version 7.7.4 not yet released.
%\item Condor version 7.7.4 released on Month Date, 2011.

\end{itemize}


\noindent New Features:

\begin{itemize}

\item None.

\end{itemize}

\noindent Configuration Variable and ClassAd Attribute Additions and Changes:

\begin{itemize}

\item None.

\end{itemize}

\noindent Bugs Fixed:

\begin{itemize}

\item None.

\end{itemize}

\noindent Known Bugs:

\begin{itemize}

\item None.

\end{itemize}

\noindent Additions and Changes to the Manual:

\begin{itemize}

\item None.

\end{itemize}


%%%%%%%%%%%%%%%%%%%%%%%%%%%%%%%%%%%%%%%%%%%%%%%%%%%%%%%%%%%%%%%%%%%%%%
\subsection*{\label{sec:New-7-7-3}Version 7.7.3}
%%%%%%%%%%%%%%%%%%%%%%%%%%%%%%%%%%%%%%%%%%%%%%%%%%%%%%%%%%%%%%%%%%%%%%

\noindent Release Notes:

\begin{itemize}

\item Condor version 7.7.3 not yet released.
%\item Condor version 7.7.3 released on Month Date, 2011.

\item
\emph{Condor now dynamically links with the ClassAds, Globus and VOMS
libraries on Mac OS X.}
A copy of these libraries is included with Condor.
\Ticket{2482}

\end{itemize}


\noindent New Features:

\begin{itemize}

\item In Condor version 7.7.2, multiple Condor installations led to the
possibility for some installations to use the wrong version of the ClassAds 
library.
This should no longer be an issue, 
as the binaries now use \Env{RUNPATH} instead of \Env{RPATH}, 
allowing use of the \Env{LD\_LIBRARY\_PATH} environment variable 
to set where to look for the shared libraries.
\Ticket{2539}

\item The Amazon SOAP interface is no longer present or supported in Condor.
The EC2 REST interface is favored and supported in Condor
using a \SubmitCmd{grid\_resource} of \SubmitCmd{ec2}.
\Ticket{2523}

\item The new \Condor{gather\_info} tool introduced in 
Condor version 7.5.6 has been updated and enhanced.
It collects data about a Condor installation, and, if desired, 
about a specific job. 
This information is useful to Condor developers to help 
debug problems in a pool or with a job.
\Ticket{1664}
\Ticket{2372}

\item The \Condor{userprio} tool supports two new command line options.
The \Opt{-grouporder} flag displays submitter entries 
for accounting groups at top of the list,
 in breadth-first order by group hierarchy.
The \Opt{-grouprollup} flag reports accounting statistics for groups 
as summed at a level within the group hierarchy.
\Ticket{1926}

\item The \Condor{collector} now avoids the performance problems caused
previously when clients initiated communication with the \Condor{collector},
but then delayed sending input.
\Ticket{2506}

\item When using versions of \Prog{glexec} that create a copy of the proxy 
for use by the job, 
Condor now ensures that this copy of the proxy is cleaned up
when the job is done.
\Ticket{2501}

\item The \Condor{startd} now logs a clear message, if it rejects a job
because no valid \Condor{starter} daemons were detected.
\Ticket{2470}

\item The new submit command \SubmitCmd{want\_graceful\_removal}
may be used to specify that a job being removed or put on hold should
be shut down gracefully, rather than being immediately hard-killed.
This allows the job to perform some final actions such as cleaning
up or saving state.  The usual policies governing the Preempting/Vacating
state apply in this case.  

This new submit command replaces a different mechanism that was added 
in Condor version 7.5.2 to achieve some of the same effects.  
The version 7.5.2 mechanism applied to vanilla jobs under Linux;
if the job set \SubmitCmd{remove\_kill\_sig} or \SubmitCmd{kill\_sig},
the hard-kill signal that Condor would normally send to end the job was
replaced with the signal specified by the user.  

With the new submit command, the version 7.5.2 mechanism is no longer used.
The soft-kill signal may still be customized using
\SubmitCmd{kill\_sig}, so a similar effect can be achieved by setting
\Expr{want\_graceful\_removal=True} and setting \SubmitCmd{kill\_sig}
to an alternative signal, if desired.  The new mechanism works on all
platforms and works for all universes in which the job is managed by
the \Condor{startd}; as such the new mechanism is not supported
in the grid, local, or scheduler universes.

In addition, the new submit command \SubmitCmd{job\_max\_vacate\_time}
replaces the \SubmitCmd{kill\_sig\_timeout} command.
\SubmitCmd{job\_max\_vacate\_time}
adjusts the time given to an evicted job for gracefully shutting down.
\Ticket{2536}

\item The \Condor{master} now logs a more informative error message
when it fails to start a daemon.
\Ticket{2580}

\item The \Condor{schedd} daemon now logs a more informative error message
when it rejects job ClassAd updates from the \Condor{shadow} due to
authorization problems.
\Ticket{2581}

\end{itemize}

\noindent Configuration Variable and ClassAd Attribute Additions and Changes:

\begin{itemize}

\item The new configuration variable \Macro{MachineMaxVacateTime} is
now used to express the maximum time in seconds that the machine is
willing to wait for a job to gracefully shut down.  
The default is 600 seconds (10 minutes).  
The boolean \MacroNI{KILL} expression was
previously used to terminate the graceful shutdown of jobs.  
It should normally be set to \Expr{False} now.  If desired, it may be
used to abort the graceful shutdown of the job earlier than
\MacroNI{MachineMaxVacateTime}.
\Ticket{2536}

\item The new configuration variable \Macro{NEGOTIATOR\_SLOT\_CONSTRAINT} 
defines an expression which constrains which ClassAds are fetched
by the \Condor{negotiator} from the \Condor{collector}
for the negotiation cycle. 
\Ticket{2277}

\item The new configuration variable 
\Macro{NEGOTIATOR\_SLOT\_POOLSIZE\_CONSTRAINT} 
replaces \Macro{GROUP\_DYNAMIC\_MACH\_CONSTRAINT}.
\MacroNI{GROUP\_DYNAMIC\_MACH\_CONSTRAINT} may still be used,
but a warning is written to the log,
identifying that the configuration needs to be updated to use the new name.
The pool size resulting from applying this constraint is used
to determine quotas for both dynamic quotas in hierarchical groups,
and when there are no groups.
\Ticket{2277}

\item The configuration variable \Macro{NEGOTIATOR\_STARTD\_CONSTRAINT\_REMOVE} 
was introduced in Condor version 7.7.1.
It has now been removed, as its functionality 
was made obsolete by \MacroNI{NEGOTIATOR\_SLOT\_CONSTRAINT}.
\Ticket{2277}

\item The configuration variables \Macro{IGNORE\_NFS\_LOCK\_ERRORS}
and \Macro{BIND\_ALL\_INTERFACES} no longer support the undocumented use of
'Y' or 'y' to mean \Expr{True}.

\end{itemize}

\noindent Bugs Fixed:

\begin{itemize}

\item When the \Condor{procd}'s named command pipe is removed, 
or when the inode of the pipe has been changed while the daemon is running, 
the \Condor{procd} will now exit.
Its previous behavior had the \Condor{procd} continue to execute 
in a useless mode of operation, unable to receive any communication.
\Ticket{2500}

\item For Mac OS X platforms, 
improper detection of a non existent process led to lines such as
\begin{verbatim}
ProcAPI sanity failure on pid 1317, age = -1901476270
\end{verbatim}
appearing in the \Condor{master} daemon log.
This should no longer be the case.
\Ticket{2594}

\item Fixed a bug introduced with hierarchical group quotas that
failed to correctly initialize table entries.
The fix adds logic to the accounting mechanism in the
\Condor{negotiator} daemon,
such that initialization occurs correctly 
when starting up and upon reconfiguration.
\Ticket{2509}

\item When \Condor{advertise} is used with the \Opt{-tcp} option, this
used to cause the following log message to appear in the \Condor{collector}
log:
\begin{verbatim}
DaemonCore: Can't receive command request from IP (perhaps a timeout?)
\end{verbatim}
\Ticket{2483}

\item Fixed a bug introduced in Condor version 7.7.0,
in which the setting of \MacroNI{NETWORK\_INTERFACE} did not have any effect.
\Ticket{2513}

\item \Prog{glexec} now also works when Condor is running as root.
\Ticket{2503}

\item The \Condor{master} daemon now successfully advertises itself in 
a Personal Condor installation,
when the \Condor{collector} is configured to use port 0
and to operate through a shared port.
\Ticket{2555}

\item Since Condor version 7.7.1, 
the configuration variable \Macro{WANT\_HOLD} did not work,
unless \Macro{WANT\_HOLD\_SUBCODE} was set to a non-zero value.
\Ticket{2565}

\item Since Condor version 7.7.2, there was a rare condition that could cause
a job to be removed from the queue,
if the job was put on hold while it was running.
In such cases, there was also a spurious
unsuspend event logged in the job's user log.
\Ticket{2577}

\item Fixed a bug introduced in 7.7.2 by the change of \Macro{OpSys} to "WINDOWS".
Submit files that used version 1 syntax for the \SubmitCmd{environment} command
were using Unix syntax rather than Windows syntax.
\Ticket{2607}

\end{itemize}

\noindent Known Bugs:

\begin{itemize}

\item None.

\end{itemize}

\noindent Additions and Changes to the Manual:

\begin{itemize}

\item None.

\end{itemize}


%%%%%%%%%%%%%%%%%%%%%%%%%%%%%%%%%%%%%%%%%%%%%%%%%%%%%%%%%%%%%%%%%%%%%%
\subsection*{\label{sec:New-7-7-2}Version 7.7.2}
%%%%%%%%%%%%%%%%%%%%%%%%%%%%%%%%%%%%%%%%%%%%%%%%%%%%%%%%%%%%%%%%%%%%%%

\noindent Release Notes:

\begin{itemize}

\item Condor version 7.7.2 released on October 11, 2011.
This release contains all features and bug fixes from Condor version 7.6.4
as are currently documented (section~\ref{sec:New-7-6-4}) in this manual. 

\item
\emph{Condor now dynamically links with the ClassAds, Globus and VOMS libraries on
linux.}
A copy of these libraries is included with Condor, under
\File{lib/condor/} in the tarball releases and under
\File{/usr/lib/condor/} or \File{/yrs/lib64/condor/} in the native package
releases.
\Ticket{2389}
\Ticket{2390}

\end{itemize}


\noindent New Features:

\begin{itemize}

\item Condor's standard universe now supports reading from and writing to
files that are larger than 2 GBytes,
when the standard universe application and
the \Condor{shadow} daemon are both 64-bit executables.
\Ticket{2337}

\item There is command line support to both suspend and continue jobs. 
The new tools \Condor{suspend} and \Condor{continue} will 
suspend and continue running jobs.
\Ticket{2368}

\item The EC2 GAHP now supports X.509 for connecting to and authenticating
with EC2 services.  See section~\ref{sec:Amazon-submit} for details
on using the X.509 protocol.
\Ticket{2084}

\item Previously, the dedicated scheduler attempted to change the
\Attr{Scheduler} attribute on all parallel job processes in a durable fashion,
resulting in an \Procedure{fsync} for each process.
This has been changed to be not durable, 
thereby improving the scalability by reducing the 
number of \Procedure{fsync} calls without impacting correctness. 
\Ticket{2367}

\item In PrivSep mode, when an error is encountered when trying to
switch to the user account chosen for running the job, 
the error message has been improved to make debugging easier.  
Now, the error message distinguishes between safety check failures 
for the UID, tracking group ID, primary group ID, and supplementary group IDs.
\Ticket{2364}

\item The name of the user used to execute the job is now logged in
the \Condor{starter} log, except when using \Prog{glexec}.
\Ticket{2268}

\item \Condor{dagman} now defaults to writing a partial DAG file
for a Rescue DAG,
as opposed to a full DAG file.
The Rescue DAG file is parsed in combination with the original DAG file, 
meaning that any
changes to the original DAG input file take effect when running a Rescue DAG.
\Ticket{2165}

\item The behavior of DAGMan is changed, such that, by default, 
POST scripts will be run regardless of the return value from 
the PRE script of the same node as described in section~\ref{dagman:SCRIPT}.  
The previous behavior of not running the POST script can be restored by
either adding the \Opt{-AlwaysRunPost} option to the \Condor{submit\_dag}
command line, 
or by setting the new configuration variable
\Macro{DAGMAN\_ALWAYS\_RUN\_POST} to \Expr{False}, 
as defined at~\ref{param:DAGmanAlwaysRunPost}.
\Ticket{2057}

\item DAGMan will now copy PRIORITY values from the DAG input file to 
the \Attr{JobPrio} attribute in the job ClassAd.  
Furthermore, the PRIORITY values are propagated to child nodes and SUBDAGs, 
so that child nodes always have priority at least that
of the maximum of the priorities of its parents.  
This has been a cause of confusion for DAGMan users.
\Ticket{2167}

% moved to 7.7.2 
% gittrac #659 
%\item Filip Krikava supplied a patch that limits the number of 
%file descriptors that DAGMan has open at a time.
%The reason for creating this capability is that
%DAGman tends to fail on wide DAGs, where many jobs are independent,
%rather than being linear, where jobs have many dependencies.

\item A matchmaking optimization has significantly improved the speed 
of matching,
when there are machines with many slots.
\Ticket{2403}

\item When the \Condor{schedd} is starting up and it encounters corruption
in its job transaction log, the error message in the log file now reports
the offset within the file at which the error occurred.
\Ticket{2450}

\end{itemize}

\noindent Configuration Variable and ClassAd Attribute Additions and Changes:

\begin{itemize}

\item The new job ClassAd attribute \Attr{PreserveRelativeExecutable}, 
when \Expr{True} prevents the \Condor{starter} from 
prepending \Attr{Iwd} to the command executable \Attr{Cmd},
when \Attr{Cmd} is a relative path name and \Attr{TransferExecutable} 
is \Expr{False}.
\Ticket{2460}

\item Attributes have been added to all daemons to publish statistics 
about the the number of timers, signals, socket, and pipe messages 
that have been handled, as well as the amount of time spent handling them.	Statistics attributes for DaemonCore
have names that begin with \Expr{DC} or \Expr{RecentDC}.
\Ticket{2354}

\item The default value of \Attr{OpSys} on Windows machines has been changed
to \AdStr{WINDOWS}, and a new attribute \Attr{OpSysVer} has been added 
that contains the version number of the operating system.  
This behavior is controlled by a new configuration variable
\Macro{ENABLE\_VERSIONED\_OPSYS} which defaults to \Expr{False} on Windows 
and to \Expr{True} on other platforms.  
The new machine ClassAd attribute \Attr{OpSys\_And\_Ver} will always contain 
the versioned operating system.  
Note that this change could cause problems with mixed pools,
because Condor version 7.7.2 \Condor{submit} may add \Expr{OpSys="WINDOWS"}, 
but machines running Condor versions prior to 7.7.2 will be publishing 
a versioned \Attr{OpSys} value,
unless there is an override in the configuration.
\Ticket{2366}

\item Configuration variable \Macro{COLLECTOR\_ADDRESS\_FILE} is now set 
in the example configuration,
similar to \MacroNI{MASTER\_ADDRESS\_FILE}.
This configuration variable is required when \Macro{COLLECTOR\_HOST} 
has the port set to 0, which means to select any available port.
In other environments, it should have no visible impact.
\Ticket{2375}

% gittrac #2197
\item Attributes have been added to the \Condor{schedd} 
to publish aggregate statistics
about jobs that are running and have completed, as well as counts of various
failures. 
% Next sentence is made into a comment, as there is no documentation
%     to look at.
% See section ??? for details.
\Ticket{2197}

\item The new configuration variable \Macro{DAGMAN\_WRITE\_PARTIAL\_RESCUE}
enables the new feature of writing a partial DAG file, instead of a full
DAG input file, as a Rescue DAG.  
See section~\ref{param:DAGManWritePartialRescue} for a definition.
Also, the configuration variable
\Macro{DAGMAN\_OLD\_RESCUE} no longer exists,
as it is incompatible with the implementation of partial Rescue DAGs.
\Ticket{2165}

\end{itemize}

\noindent Bugs Fixed:

\begin{itemize}

\item Fixed a bug introduced in Condor version 7.7.1, 
in the standard universe,
where the \Syscall{getdirentries} call failed during remote I/O situations.
\Ticket{2467}

\item Fixed a bug in the \Condor{startd} that was preventing dynamic slots
from being properly instantiated from partitionable slots.
\Ticket{2507}

\item Fixed a bug introduced in Condor version 7.7.0, 
in which the \Condor{startd} may erroneously report 
\Expr{Can't find hostname of client machine.}
In cases where Condor was unable to identify the host name, 
the \Attr{ClientMachine}
attribute in the machine ClassAd would have gone unset.
\Ticket{2382}

\item Fixed a bug existing since April 2001,
in which on start up of the \Condor{schedd}, with parallel universe jobs, 
the job queue sanity checking code would change the \Attr{Scheduler}
attribute on jobs,
only to have the attribute changed later by the dedicated scheduler.
\Ticket{2367}

\item Machine ClassAds with the \Attr{Offline} attribute set to \Expr{True},
and  with neither \Attr{MyType} nor \Attr{TargetType} 
attributes defined caused
the \Condor{collector} to fail to start when it was next restarted.
\Ticket{2417}

\item Fixed a file descriptor leak in the EC2 GAHP,
which would cause grid-type ec2 jobs to become held. 
The \Attr{HoldReason} for most such jobs would be 
\Expr{Unable to read from accesskey file.}
\Ticket{2447}

\item Fixed a bug that could cause a job's standard output and error to
be written to the wrong location when \SubmitCmd{should\_transfer\_files} was
set to \Expr{IF\_NEEDED},
and the job runs on the machine where file transfer is not needed.
If the standard output or error file names contained any path information,
the output would be written to \File{\_condor\_stdout} or
\File{\_condor\_stderr} in the job's initial working directory.
\Ticket{1811}

\item Fixed a bug introduced in Condor version 7.7.1
that could cause the \Condor{schedd} daemon to crash after
failing to expand a \verb@$$@ macro in the job ClassAd.
\Ticket{2491}

\end{itemize}

\noindent Known Bugs:

\begin{itemize}

\item In Condor version 7.7.2, 
the Condor daemons on Linux platforms rely on shared libraries.  
A bug in Condor version 7.7.1 and all previous versions of Condor
prevents a 7.7.1 \Condor{master} from starting 7.7.2 or later daemons.
This also means that a 7.7.1 \Condor{master} cannot upgrade itself to 
version 7.7.2.  
If a 7.7.1 \Condor{master} binary is replaced with 
a 7.7.2 \Condor{master} binary, 
Condor will shut off, and need to be restarted by hand.

\end{itemize}

\noindent Additions and Changes to the Manual:

\begin{itemize}

\item None.

\end{itemize}


%%%%%%%%%%%%%%%%%%%%%%%%%%%%%%%%%%%%%%%%%%%%%%%%%%%%%%%%%%%%%%%%%%%%%%
\subsection*{\label{sec:New-7-7-1}Version 7.7.1}
%%%%%%%%%%%%%%%%%%%%%%%%%%%%%%%%%%%%%%%%%%%%%%%%%%%%%%%%%%%%%%%%%%%%%%

\noindent Release Notes:

\begin{itemize}

%\item Condor version 7.7.1 not yet released.
\item Condor version 7.7.1 released on September 12, 2011.
This developer release contains all bug fixes from Condor version 7.6.3.

\end{itemize}


\noindent New Features:

\begin{itemize}

\item
\emph{Condor now dynamically links with the OpenSSL and Kerberos security
libraries, and Condor will use the operating system's version of these
libraries,  when they are available.} 
The tarball release of Condor on Linux platforms includes 
a copy of these libraries.  
If the operating system's version is incompatible with Condor, 
Condor will use its own copy instead.
Condor's copy of these libraries is located under \File{lib/condor/}.
To prevent Condor from considering using them, delete these libraries.
\Ticket{1874}

\item 
The ClassAd language now has an \Procedure{unparse} function.  
It converts an expression into a string, 
which is handy with the new \Procedure{eval} function.
\Ticket{1613}

\item
The new job ClassAd attribute \Attr{KeepClaimIdle} is defined with an integer
number of seconds in the job submit description file, as the example:
\begin{verbatim}
  +KeepClaimIdle = 300
\end{verbatim}
If set, then when the job exits, 
if there are no other jobs immediately ready to run for this user, 
the \Condor{schedd} daemon,
instead of relinquishing the claim back to the \Condor{negotiator}, 
will keep the claim for the specified number of seconds.  
This is useful if another job will be arriving soon, 
which can happen with linear DAGs.  
The \Condor{startd} slot
will go to the Claimed Idle state for at least that many seconds until
either a new job arrives or the timeout occurs.
See page~\pageref{sec:Job-ClassAd-Attributes},
the unnumbered Appendix A for a complete definition of this
job ClassAd attribute.
\Ticket{2094}

% gittrac #2122
\item The new \Arg{PRE\_SKIP} key word in DAGMan changes the
behavior of DAG node execution such that the node's job and POST script
may be skipped based on the exit value of the PRE script.
See section ~\ref{dagman:SCRIPT} for details.
\Ticket{2122}

% uncomment item, if it appears in 7.7.1
% gittrac #659 
%\item Filip Krikava supplied a patch that limits the number of 
%file descriptors that DAGMan has open at a time.
%The reason for creating this capability is that
%DAGman tends to fail on wide DAGs, where many jobs are independent,
%rather than being linear, where jobs have many dependencies.

\end{itemize}

\noindent Configuration Variable and ClassAd Attribute Additions and Changes:

\begin{itemize}

\item The new configuration variable 
\Macro{NEGOTIATOR\_STARTD\_CONSTRAINT\_REMOVE} defaults to \Expr{False}.
When \Expr{True}, any ClassAds not satisfying the expression 
in \MacroNI{GROUP\_DYNAMIC\_MACH\_CONSTRAINT} are removed from the
list of \Condor{startd} ClassAds considered for negotiation.
\Ticket{2232}

\item The new configuration variable
\Macro{NEGOTIATOR\_UPDATE\_AFTER\_CYCLE} defaults to \Expr{False}.
When \Expr{True}, it forces the \Condor{negotiator} daemon
to update the negotiator ClassAd in the \Condor{collector} daemon
at the end of every negotiation cycle.  
This is handy for monitoring and debugging activities.
\Ticket{2373}

\end{itemize}

\noindent Bugs Fixed:

\begin{itemize}

\item Expressions for periodic policies such as 
\MacroNI{PERIODIC\_HOLD} and \MacroNI{PERIODIC\_RELEASE} 
could inadvertently cause a claim to be released,
 if the \Condor{shadow} exited before waiting for final update from the 
\Condor{starter}. 
\Ticket{2329}

\item \Condor{submit} previously could incorrectly detect references
in the requirements expression to special attributes such as
\Attr{Memory} when the name of the attribute happened to appear in a
string literal or as part of the name of some other attribute.  
The detection of references to various special attributes influences the
automatic requirements which are appended to the job requirements.
\Ticket{2350}

\item In rare cases, CCB requests could cause the server to hang for
20 seconds while waiting for all of the request to arrive.
\Ticket{2360}

\end{itemize}

\noindent Known Bugs:

\begin{itemize}

\item None.

\end{itemize}

\noindent Additions and Changes to the Manual:

\begin{itemize}

\item None.

\end{itemize}


%%%%%%%%%%%%%%%%%%%%%%%%%%%%%%%%%%%%%%%%%%%%%%%%%%%%%%%%%%%%%%%%%%%%%%
\subsection*{\label{sec:New-7-7-0}Version 7.7.0}
%%%%%%%%%%%%%%%%%%%%%%%%%%%%%%%%%%%%%%%%%%%%%%%%%%%%%%%%%%%%%%%%%%%%%%

\noindent Release Notes:

\begin{itemize}

\item Condor version 7.7.0 released on July 29, 2011.
This developer release contains all bug fixes from Condor version 7.6.2.

\end{itemize}


\noindent New Features:

\begin{itemize}

\item A full port of Condor is available for RedHat Enterprise Linux 6
on the x86\_64 processor.
A full port includes support for the standard universe.

\item The matchmaking attributes \Attr{SubmitterUserResourcesInUse}
and \Attr{RemoteUserResourcesInUse} are now biased by slot weights.

% gittrac #1971
\item \Condor{submit} now accepts the new command line option \Opt{-addr},
naming the IP address of the \Condor{schedd} to submit to.

\item The \Condor{vm\_gahp} now is dynamically linked to libvirt.  
We believe this makes it more portable.

\item Programs \Condor{reconfig\_schedd} and \Condor{master\_off}
are no longer part of the distribution.
These programs were replaced many years ago by the more general
\Condor{reconfig} and \Condor{off} commands.

\item On Windows platforms, improved the ability of the \Condor{starter}
and \Condor{shadow} daemons to clean up the execute directory,
if jobs have changed the ACLs or permissions on files they have created.

\item \Condor{submit} now sets a default value for job ClassAd attribute
\Attr{RequestMemory}.

\item The submission performance of cream grid jobs has been substantially
improved by batching submit requests.

\item \Condor{q} \Opt{-better} now has cleaner output, and informs
the user when negotiation has not yet occurred.

\item Implemented many improvements to the Condor \Prog{init} scripts.

\item Deltacloud support has been updated to deltacloud version 0.8.

% gittrac #1960
\item As of Condor version 7.6.0,
vm universe submit description files no longer support
automatic creation of cdrom images from text input file.
Users must now explicitly create ISO images and transfer them
with the job.

\item \Condor{q} now supports the new option \Opt{-stream-results}.
  When this option is specified, \Condor{q} displays results as they
  are fetched from the job queue, rather than buffering up the query
  results before displaying anything.

% gittrac #1871 
% gittrac #2295
\item The new submit description file command \SubmitCmd{stack\_size} 
  applies to Linux jobs that are not running in the standard universe. 
  It sets the allocation of stack space to be other than the default
  value, which is unlimited.
  It also advertises the job ClassAd attribute \AdAttr{StackSize}.

% gittrac #1550
\item The new ClassAd function \Code{stringListsIntersect} evaluates to 
  \Expr{True} if two strings of delimited elements have any matching elements,
  and it evaluates to \Expr{False} otherwise.

% gittrac #1821
\item The grid universe now supports the \SubmitCmd{ec2} resource type,
  which uses the EC2 Query (REST) API to start virtual machines on cloud
  resources.

% gittrac #2090 
\item The behavior of DAGMan has changed, 
such that if multiple definitions of a VARS macroname 
for a specific node within a DAG input exist,
a warning is written to the log, of the format
\begin{verbatim}
Warning: VAR <macroname> is already defined in job <JobName>
Discovered at file "<DAG input file name>", line <line number>
\end{verbatim}
See section ~\ref{dagman:VARS} for details.

% gittrac #2297
\item The version number for ClassAds now matches the Condor version number. 

% gittrac #2259
\item When \Prog{glexec} fails to execute a job,
diagnostic error messages produced by \Prog{glexec} used to be discarded.
These error messages are now displayed in the log of the \Condor{starter} 
and in the job's hold reason. 

% gittrac #2185
\item New submit description file commands
\SubmitCmd{periodic\_hold\_reason}, \SubmitCmd{periodic\_hold\_subcode},
\SubmitCmd{on\_exit\_hold\_reason}, and \SubmitCmd{on\_exit\_hold\_subcode}
permit the job to set a hold reason string and subcode number.
Similarly, the system job policy can specify the reason and subcode 
using \Macro{SYSTEM\_PERIODIC\_HOLD\_REASON} and 
\Macro{SYSTEM\_PERIODIC\_HOLD\_SUBCODE}.
In addition, the \Condor{hold} command now accepts a \Opt{-subcode} option,
which is used to set the job attribute \Attr{HoldReasonSubCode}. 

\item If the \Condor{shadow} cannot write to the user log, 
the job is now put on hold.

\end{itemize}


\noindent Configuration Variable and ClassAd Attribute Additions and Changes:

\begin{itemize}

\item The new configuration variable \Macro{NEGOTIATOR\_UPDATE\_AFTER\_CYCLE}
defaults to \Expr{False}.
If set to \Expr{True}, it will force the \Condor{negotiator} daemon
to publish an update ClassAd to the \Condor{collector} at the end of 
every negotiation cycle. 
This is useful if monitoring cycle-based statistics.

\item The configuration variables for security 
\Macro{DENY\_CLIENT} and \Macro{HOSTDENY\_CLIENT}
now also look for the prefixes \Expr{TOOL} and \Expr{SUBMIT}.
 
% gittrac #1249
\item \Macro{CONDOR\_VIEW\_HOST} is now a comma and/or white space separated
list of hosts, in order to support more than one CondorView host.

\item For a job with an X.509 proxy credential, the new job ClassAd
attribute \AdAttr{X509UserProxyEmail} is the email address extracted
from the proxy.

% gittrac 2067
\item On Linux execute machines with kernel version more recent than 2.6.27,
the proportional set size (PSS) in Kbytes summed across all
processes in the job is now reported in the attribute
\AdAttr{ProportionalSetSizeKb}.  If the execute machine does not
support monitoring of PSS or PSS has not yet been measured, this
attribute will be undefined.  PSS differs from \AdAttr{ImageSize} in
how memory shared between processes is accounted.  The PSS for one
process is the sum of that process' memory pages divided by the
number of processes sharing each of the pages.  \AdAttr{ImageSize} is
the same, except there is no division by the number of processes
sharing the pages.

% gittrac #1755
\item The new configuration variable \Macro{DAGMAN\_USE\_STRICT} 
turns warnings into errors, as defined in section~\ref{param:DAGManUseStrict}.

% gittrac #2006
\item The \Condor{schedd} now publishes performance-related statistics.
  Page~\pageref{sec:Scheduler-ClassAd-Attributes} in Appendix A contains
  definitions for these new attributes:
  \begin{itemize}
    \item \Attr{DetectedMemory}
    \item \Attr{DetectedCpus}
    \item \Attr{UpdateInterval}
    \item \Attr{WindowedStatWidth}
    \item \Attr{ExitCode<N>}
    \item \Attr{ExitCodeCumulative<N>}
    \item \Attr{JobsSubmitted}
    \item \Attr{JobsSubmittedCumulative}
    \item \Attr{JobsStarted}
    \item \Attr{JobsStartedCumulative}
    \item \Attr{JobsCompleted}
    \item \Attr{JobsCompletedCumulative}
    \item \Attr{JobsExited}
    \item \Attr{JobsExitedCumulative}
    \item \Attr{ShadowExceptions}
    \item \Attr{ShadowExceptionsCumulative}
    \item \Attr{JobSubmissionRate}
    \item \Attr{JobStartRate}
    \item \Attr{JobCompletionRate}
    \item \Attr{MeanTimeToStart}
    \item \Attr{MeanTimeToStartCumulative}
    \item \Attr{MeanRunningTime}
    \item \Attr{MeanRunningTimeCumulative}
    \item \Attr{SumTimeToStartCumulative}
    \item \Attr{SumRunningTimeCumulative}
  \end{itemize}

% gittrac #1930
\item For Windows platforms, the \Condor{startd} now publishes the 
ClassAd attribute \Attr{DotNetVersions},
containing a comma separated list of installed .NET versions.

\end{itemize}

\noindent Bugs Fixed:

\begin{itemize}

\item Fixed a bug in which the \Condor{startd} daemon can get stuck in a
loop trying to execute an invalid, 
that is non-existent, Daemon ClassAd Hook job.

\item Fixed bug that would cause the \Condor{startd} daemon to incorrectly
report Benchmarking activity instead of Idle activity,
when there is a problem launching the benchmarking programs.

\item On Windows only, fixed a rare bug that could cause
a sporadic access violation when a Condor daemon spawned another process.

\item Fixed a bug introduced in Condor version 7.5.5,
which caused the \Condor{schedd} to die managing parallel jobs.

% commented out, as this bug fix is listed in the 7.6.1 version history.
% \item Fixed bug present throughout ClassAds,
% where expressions expecting a floating point value returned an error,
% if they got a boolean value.  This is common in \MacroNI{RANK} expressions.

\item The \Condor{startd} daemon now looks up the \Condor{kbdd} daemon address
on every update.  
This fixed problems if the \Condor{kbdd} daemon is restarted 
during the \Condor{startd} lifespan.

\item Fixed bug in \Condor{hold} that happened if the hold
reason contained a double quote character.

\item Fixed a bug introduced in Condor version 7.5.6 that
caused any Daemon ClassAd hook job with non-empty value for
\MacroNI{STARTD\_CRON\_<JobName>\_ARGS},
\MacroNI{SCHEDD\_CRON\_<JobName>\_ARGS}
or \MacroNI{BENCHMARKS\_<JobName>\_ARGS} to fail.
Also, the specification of 
\MacroNI{STARTD\_CRON\_<JobName>\_ENV},
\MacroNI{SCHEDD\_CRON\_<JobName>\_ENV},
or \MacroNI{BENCHMARKS\_<JobName>\_ENV} for these jobs was ignored.

\item Fixed bug in the RPM \Prog{init} script. 
A status request would always report Condor as inactive, 
and a shutdown request would not report failure if there was a
timeout shutting down Condor.

\item File transfer plug-ins now have a correctly set environment.

\item Fixed a problem with detecting IBM Java Virtual Machines whose
version strings have embedded newline characters.

\item \Condor{q} \Opt{-analyze} now works with ClassAd built-in functions.

\item Fixed bug in \Condor{q} \Opt{-run}, such that it displays
the host name correctly for local and scheduler universe jobs.

\item Standalone checkpointing now works with compressed checkpoints again.
This had been broken in Condor version 7.5.4.

%gittrac 1962
\item On Windows, \Prog{net stop condor} would sometimes cause the
\Condor{master} daemon to crash.  This is now fixed.

% gittrac #1928
\item \AdAttr{JobUniverse} was effectively a required attribute for
  jobs created via the Fetch Work hook,
  due to the need to set the \MacroNI{IS\_VALID\_CHECKPOINT\_PLATFORM}
  expression, such that it would not evaluate to \Expr{Undefined}.
  Now the default \MacroNI{IS\_VALID\_CHECKPOINT\_PLATFORM} expression
  evaluates to \Expr{True} when \AdAttr{JobUniverse} is not defined.

% gittrac #1943
\item When there are multiple cpus but only one slot, the slot name no
longer begins with \Expr{slot1@}.

% gittrac #1805 
\item The tool \Condor{advertise} seemed to be trying too hard to resolve
host names. This was fixed to only do the minimally necessary 
number of look ups.

\end{itemize}

\noindent Known Bugs:

\begin{itemize}

\item None.

\end{itemize}

\noindent Additions and Changes to the Manual:

\begin{itemize}

\item None.

\end{itemize}


%%%      PLEASE RUN A SPELL CHECKER BEFORE COMMITTING YOUR CHANGES!
%%%      PLEASE RUN A SPELL CHECKER BEFORE COMMITTING YOUR CHANGES!
%%%      PLEASE RUN A SPELL CHECKER BEFORE COMMITTING YOUR CHANGES!
%%%      PLEASE RUN A SPELL CHECKER BEFORE COMMITTING YOUR CHANGES!
%%%      PLEASE RUN A SPELL CHECKER BEFORE COMMITTING YOUR CHANGES!

%%%%%%%%%%%%%%%%%%%%%%%%%%%%%%%%%%%%%%%%%%%%%%%%%%%%%%%%%%%%%%%%%%%%%%
\section{\label{sec:History-7-6}Stable Release Series 7.6}
%%%%%%%%%%%%%%%%%%%%%%%%%%%%%%%%%%%%%%%%%%%%%%%%%%%%%%%%%%%%%%%%%%%%%%

This is a stable release series of Condor.
As usual, only bug fixes (and potentially, ports to new platforms)
will be provided in future 7.6.x releases.
New features will be added in the 7.7.x development series.

The details of each version are described below.

%%%%%%%%%%%%%%%%%%%%%%%%%%%%%%%%%%%%%%%%%%%%%%%%%%%%%%%%%%%%%%%%%%%%%%
\subsection*{\label{sec:New-7-6-7}Version 7.6.7}
%%%%%%%%%%%%%%%%%%%%%%%%%%%%%%%%%%%%%%%%%%%%%%%%%%%%%%%%%%%%%%%%%%%%%%

\noindent Release Notes:

\begin{itemize}

\item Condor version 7.6.7 not yet released.
%\item Condor version 7.6.7 released on Month Date, 2012.

\end{itemize}


\noindent New Features:

\begin{itemize}

\item None.

\end{itemize}

\noindent Configuration Variable and ClassAd Attribute Additions and Changes:

\begin{itemize}

\item None.

\end{itemize}

\noindent Bugs Fixed:

\begin{itemize}

\item Added the ability to delay reconfig in daemon core, and applied to the
\Condor{negotiator} to defer reconfiguration requests during the negotiation
cycle until after the cycle completes.
\Ticket{2931}

\item Fixed a potential infinite loop in the \Condor{gridmanager} for gt2
grid universe jobs. If the GRAM jobmanager was listening on a different
port than the \Condor{gridmanager} expected, the \Condor{gridmanager}
would alternate between states GM\_REGISTER and GM\_RESTART,
as visible in the \Condor{gridmanager} daemon log.
\Ticket{2916}

\item Added logic to the \Condor{negotiator} that enables job preemption 
to properly respect hierarchical group quotas.
\Ticket{2570}

\item Added logic to the \Condor{negotiator} to negotiate with full group 
quota instead of allocated-slots when no surplus sharing is in effect, to
address a bug where groups could fail to claim weighted slots.
\Ticket{2958}

\item Fixed a bug introduced in Condor version 7.6.2 that affects jobs run via
\Prog{glexec}.  
When \Prog{glexec} was configured in log-only mode, 
Condor failed to execute the job,
but reported that the job exited with exit code 1.
In such cases, 
the \Expr{stderr} of the job contained the following message:
\begin{verbatim}
fdpass_recv error on new_sock_fd
\end{verbatim}
\Ticket{2840}

\item Fixed a rare bug in which if the \Condor{startd} was configured 
to use partitionable slots,
it was possible for the \Condor{startd} to get partitioned into more slots
than there were resources.
That is, it was possible for a four cpu \Condor{startd}
to split into five slots.
\Ticket{2816}

\item Fixed a rare bug seen with parallel universe jobs,
in which if a claim was removed as the \Condor{shadow} was starting up, 
all ranks of the job would never completely start.
\Ticket{2786}

\item Fixed a bug that caused disk capacity to be under-reported 
on Windows platforms for drives with 1TB or more of free space.
\Ticket{2798}

\item Fixed a bug that caused communication failure in some cases,
 after the failure of authentication, when authentication was configured
to be optional.
\Ticket{2845}

\item Fixed a bug in which if the configuration variable \Macro{EVENT\_LOG} was 
set but the defined file was not writable,
the user's job log would not be written to. 
This bug would have been observable with DAGMan.
\Ticket{2858}

\item NorduGrid ARC LDAP servers that return attributes in an unexpected
order no longer cause the \Condor{gridmanager} to exit.
\Ticket{2888}

\item Condor failed to execute jobs when using \Prog{glexec} 
versions 0.9.0 through 0.9.5.
\Ticket{2907}

\end{itemize}

\noindent Known Bugs:

\begin{itemize}

\item None.

\end{itemize}

\noindent Additions and Changes to the Manual:

\begin{itemize}

\item None.

\end{itemize}


%%%%%%%%%%%%%%%%%%%%%%%%%%%%%%%%%%%%%%%%%%%%%%%%%%%%%%%%%%%%%%%%%%%%%%
\subsection*{\label{sec:New-7-6-6}Version 7.6.6}
%%%%%%%%%%%%%%%%%%%%%%%%%%%%%%%%%%%%%%%%%%%%%%%%%%%%%%%%%%%%%%%%%%%%%%

\noindent Release Notes:

\begin{itemize}

\item Condor version 7.6.6 released on January 17, 2012.

\end{itemize}


\noindent New Features:

\begin{itemize}

\item None.

\end{itemize}

\noindent Configuration Variable and ClassAd Attribute Additions and Changes:

\begin{itemize}

\item None.

\end{itemize}

\noindent Bugs Fixed:

\begin{itemize}

\item Fixed a memory leak affecting the \Condor{schedd} when the
configuration variable
\Macro{EVENT\_LOG\_JOB\_AD\_INFORMATION\_ATTRS} and/or the submit
description file command \SubmitCmd{job\_ad\_information\_attrs} were used.
\Ticket{2730}

\item Fixed a bug in the Windows implementation of \Condor{chirp} that caused
it to always return a status of -1073740777 for \Condor{chirp}
 commands that succeeded.
\Ticket{2739}

\item Fixed a bug in the Windows implementation of \Condor{chirp} that caused
the \Opt{put}, \Opt{whoami}, \Opt{getdir}, and \Opt{fetch} 
\Condor{chirp} commands to always fail.
\Ticket{2743}

\item Fixed a bug in the checkpoint server that could cause it to abort and
crash during a file rename operation on RHEL6 and newer versions.
\Ticket{2738}

\item Fix a bug introduced in Condor version 7.6.5,
that could cause the \Condor{schedd} to exit with the following error:
\begin{verbatim}
ERROR "Send_Signal: sent unsafe pid (0)" at line 5492 in file
/home/condor/execute/dir_10444/userdir/src/condor_daemon_core.V6/daemon_core.cpp
\end{verbatim}
\Ticket{2736}

\item Fixed the example Linux startup script \File{condor.boot.rpm}
to no longer assume that the contents of file \File{/var/run} will 
persist across a reboot,
or that environment variable \Env{USER} is set.
\Ticket{2133}

\end{itemize}

\noindent Known Bugs:

\begin{itemize}

\item None.

\end{itemize}

\noindent Additions and Changes to the Manual:

\begin{itemize}

\item None.

\end{itemize}


%%%%%%%%%%%%%%%%%%%%%%%%%%%%%%%%%%%%%%%%%%%%%%%%%%%%%%%%%%%%%%%%%%%%%%
\subsection*{\label{sec:New-7-6-5}Version 7.6.5}
%%%%%%%%%%%%%%%%%%%%%%%%%%%%%%%%%%%%%%%%%%%%%%%%%%%%%%%%%%%%%%%%%%%%%%

\noindent Release Notes:

\begin{itemize}

\item Condor version 7.6.5 released on December 28, 2011.

\item Restored the semantics of \Macro{GROUP\_AUTOREGROUP} to the
behavior it exhibited before Hierarchical Group Quotas were introduced
in Condor version 7.5.6.
That behavior has submitters with no accounting group,
which are listed as \Expr{<none>}, negotiate last.
And, in addition, any accounting groups with \MacroNI{GROUP\_AUTOREGROUP}
enabled negotiate both normally and then also along with the 
submitters with no accounting group.
For Condor versions 7.5.6 through 7.6.4, configuration variable
\MacroNI{GROUP\_AUTOREGROUP} (or \MacroNI{GROUP\_AUTOREGROUP\_<groupname>})
was a synonym for 
\Macro{GROUP\_ACCEPT\_SURPLUS} 
(or \MacroNI{GROUP\_ACCEPT\_SURPLUS\_<groupname>}).
They now implement distinct features,
and it is not legal to set both to \Expr{True} in the configuration
for the \Condor{negotiator}.
\Ticket{2679}

\end{itemize}


\noindent New Features:

\begin{itemize}

\item Added explicit support for Linux kernels with a major version number of 3,
to detect and utilize the load average information.
\Ticket{2579}

\end{itemize}

\noindent Configuration Variable and ClassAd Attribute Additions and Changes:

\begin{itemize}

\item None.

\end{itemize}

\noindent Bugs Fixed:

\begin{itemize}


%\item Exposed the negotiation order of accounting groups to configuration
%via the configuration parameter \Macro{GROUP\_SORT\_EXPR}.
%\Ticket{2678}

\item Fixed a bug in Chirp when using absolute file paths. This bug caused
most MPI jobs to fail in the parallel universe.
\Ticket{2630}

\item Fixed a bug in mapping users using the \Macro{CERTIFICATE\_MAPFILE}
mechanism, where entries using the NTSSPI method on Windows would not be mapped
using the map file, but would instead fall back to just the user name.
\Ticket{2709}

\item Fixed a hierarchical accounting groups bug in which
the \Condor{schedd} did not properly restore accounting group 
information to submitters on a restart of the \Condor{schedd},
and therefore negotiated for and allocated machines incorrectly.
\Ticket{2705}

\item The Windows installer had a bad value set for the configuration variable
\Macro{JAVA\_CLASSPATH\_SEPARATOR}, causing java universe jobs to fail. 
\Ticket{2586}

\item \MacroNI{HDFS} was not listed in the default
\MacroNI{DC\_DAEMON\_LIST}, so the \Condor{hdfs} daemon exited
shortly after being started,
and the HDFS service did not run.
\Ticket{849}

\item File System (FS) authentication now works when \File{/tmp} 
is on a Btrfs file system.  Previously, authentication failed.
\Ticket{2583}

\item Fixed a bug that caused a failure to start jobs when using PrivSep
and supplemental group process tracking.  
Prior to Condor version 7.6.4, this problem
only occurred when \Macro{USE\_CLONE\_TO\_CREATE\_PROCESSES} was set
to \Expr{False}.  
In Condor version 7.6.4, the problem occurred regardless of the setting
of this configuration variable.
\Ticket{2658}

\item Fixed a performance problem on Windows platforms
that caused claim activations to
fail when more than about 8 jobs were already running on that machine.
\Ticket{2441}

\item Fixed a bug in which the submit event would not be written to the user
job log,
if the job was submitted with the \Opt{-remote} or \Opt{-spool} option to
\Condor{submit}.
\Ticket{2569}

\item Fixed a bug that caused \Condor{q} with the \Opt{-analyze} option
to fail,
if a job or a machine ClassAd contained a string attribute ending in 
a backslash.
This resulted in output of the error message
\begin{verbatim}
  Unable to process machine ClassAds
\end{verbatim}
or
\begin{verbatim}
  Unable to process job ClassAd
\end{verbatim}
\Ticket{2603}

\item Fixed a bug that caused the \Condor{startd} to crash when being
reconfigured,
if the reconfigure caused the \Condor{startd} to remove 
a running Daemon ClassAd Hook job.
\Ticket{2636}

\item Configuration variables of the form \MacroNI{MAX\_<SUBSYS>\_<LEVEL>\_LOG}
now work properly on 32-bit Linux platforms.
Previously, the corresponding log file would grow without bound.
\Ticket{2638}

\item Fixed a bug in which Condor would fail to properly detect that it 
was running as Local System for non-English versions of Windows.
The bug caused Condor to fail to run jobs on the slot accounts.
\Ticket{2642}

\item Fixed a bug in the Windows version of Condor,
in which the transfer of output failed due to the use of the Everyone account,
which lacks read permission.
Usage of the Everyone account occurred as a fallback,
when the account name failed to exist because it
included the domain of the local submit machine.
The fix adds the same capability as exists on Linux platforms,
which uses the user name without the domain.
\Ticket{2643}

\item Fixed a bug in which job submission via Condor-C could fail,
because it did not convert account names to fully qualified (including domain)
before comparing to see
if the current account was the same as the desired account.
\Ticket{2644}

\item Fixed a bug in which use of the submit command 
\SubmitCmd{transfer\_input\_files} did not work for directories on 
Windows platforms.
\Ticket{2387}

\item Fixed a bug that could cause a failure in cleaning up job processes
when using \Prog{glexec} after a restart of the \Condor{master} daemon.
\Ticket{2614}

\item Fixed a bug in \Condor{power} that caused it to fail when
operating on a machine with a 15-byte subnet mask string.
\Ticket{2651}

\item Fixed a bug that could cause the \Condor{schedd} to no longer start
idle jobs or send ClassAds to the \Condor{collector}.
\Ticket{2647}

\item Fixed a bug that could cause the \Condor{schedd} to crash if a
hold reason contained a percent character (\verb|%|),
and the user log for the job was in XML format.
\Ticket{2660}

\item Fixed a Windows 7 and Vista bug in \Condor{softkill},
in which it would fail to kill the target process,
when run by a Personal Condor inside a System Condor slot account.
\Ticket{2677}

\item A possible fix has been made for a problem in which the
CCB-enabled daemon took an unexpectedly long time to timeout when
reading from the CCB server.  Additional information is logged
to help identify the problem if it still remains.
\Ticket{2695}

\item Fixed a bug in \Condor{dagman} that occurred when 
dealing with nested splices.
\Condor{dagman} incorrectly issued a parse error and exited
in the case where the parent splice contained only splices, and no nodes jobs.
\Ticket{1751}

\item Fixed a bug that caused grid universe jobs submitted via SOAP to
be held when trying to write output files into the spool directory.
\Ticket{2568}

\item Fixed a bug that caused \Condor{credd} and possibly other
daemons to crash when the file used for \MacroNI{CERTIFICATE\_MAPFILE}
contained more than 80 entries.
\Ticket{2409}

\item Fixed a bug that caused hibernation to fail on certain Linux platforms
for certain hibernation states.
To work correctly on these Linux platforms,
the plug-in needs the command line arguments defined by
\Macro{HIBERNATION\_PLUGIN\_ARGS} when initially invoked, 
as well as for other invocations.  
\Ticket{2561}

\item The \Condor{schedd} now aborts the claim and reschedules the job,
if it does not hear from the \Condor{startd} for longer than the job
lease duration.
\Ticket{2706}

\item Fixed some bugs in the renewing of CREAM job leases. Before, the
\Condor{gridmanager} could fail to renew the leases or attempt to set
lease expirations in the past.
\Ticket{2351}
\Ticket{2455}

\end{itemize}

\noindent Known Bugs:

\begin{itemize}

\item None.

\end{itemize}

\noindent Additions and Changes to the Manual:

\begin{itemize}

\item None.

\end{itemize}


%%%%%%%%%%%%%%%%%%%%%%%%%%%%%%%%%%%%%%%%%%%%%%%%%%%%%%%%%%%%%%%%%%%%%%
\subsection*{\label{sec:New-7-6-4}Version 7.6.4}
%%%%%%%%%%%%%%%%%%%%%%%%%%%%%%%%%%%%%%%%%%%%%%%%%%%%%%%%%%%%%%%%%%%%%%

\noindent Release Notes:

\begin{itemize}

%\item Condor version 7.6.4 not yet released.
\item Condor version 7.6.4 released on October 21, 2011.

\end{itemize}


\noindent New Features:

\begin{itemize}

\item The new Windows-only \Condor{rmdir} was included in Condor version 7.6.0,
but there was no version history entry for this introduced tool at release.
This item attempts to correct that oversight, 
as well as identify that usage of \Condor{rmdir},
instead of the built-in Windows \Prog{rmdir}, 
is enabled by default.
\Condor{rmdir} worked correctly in Condor version 7.6.0, 
contained a bug in Condor version 7.6.1,
and was fixed in Condor version 7.6.2.
\Ticket{1877}


\end{itemize}

\noindent Configuration Variable and ClassAd Attribute Additions and Changes:

\begin{itemize}

\item The new configuration variable \Macro{<Keyword>\_HOOK\_JOB\_EXIT\_TIMEOUT}
defines the number of seconds that the \Condor{starter} will wait
before continuing with a shut down,
if a hook defined by \MacroNI{<Keyword>\_HOOK\_JOB\_EXIT} has not completed.
The addition of this configuration variable fixes the bug described below.
\Ticket{2543}

\item The new configuration variable \Macro{SKIP\_WINDOWS\_LOGON\_NETWORK} 
is a boolean value which specifies whether the Windows
\Expr{LOGON\_NETWORK} authentication is skipped or not.
If skipped, Condor tries \Expr{LOGON\_INTERACTIVE} authentication first.
The addition of this configuration variable fixes the bug described below.
\Ticket{2549}  

% This actually should have gone into 7.6.1, where it was committed.
% It appears here in the 7.6.4 history by wrangler's order, without
% owning up to the improper appearance in this version.
\item The new configuration variable \Macro{SHADOW\_RUN\_UNKNOWN\_USER\_JOBS} 
defaults to \Expr{False}.
When \Expr{True}, 
it allows the \Condor{shadow} daemon to run jobs remotely submitted from 
users not in the local password file.
\Ticket{2004}

\end{itemize}
\noindent Bugs Fixed:

\begin{itemize}

%\item Properly support \Attr{MAX\_<subsys>\_LOG} values >= 2GB.
\item Implemented proper support of values greater than or equal to  2 GBytes
set for the configuration variable \Macro{MAX\_<SUBSYS>\_LOG}.
\Ticket{2471}

\item Updated the \Condor{negotiator} daemon's assessment of pool size 
to properly take partitionable slots into account.
See section ~\ref{sec:Configuring-SMP} for an explanation of 
partitionable slots on SMP machines.
\Ticket{2440}

\item Provided an informative error message 
when the \Condor{userprio} tool cannot locate the \Condor{negotiator} daemon.
\Ticket{2478}

\item \Condor{userprio} and the \Condor{negotiator} daemon 
did not correctly handle the names of submitters, 
when these names exceeded 63 characters in length.
The fix handles submitter names of arbitrary length.
\Ticket{2445}

\item Removed a spurious boolean flag reset in \Condor{q},
which resulted in an order dependency between the command line arguments
\Opt{-long} and \Opt{-format}.
\Ticket{2498}

\item Fixed a bug in which a graceful shutdown of a \Condor{startd}
did not correctly handle jobs using job deferral
which have landed on an execute machine but have not yet
reached their deferral time.
These jobs would appear to be running, despite the lack of
a \Condor{starter} daemon. 
These jobs now correctly transition to the idle state.
\Ticket{2486}

\item Corrected a hierarchical group quota bug in which
the user accounting mechanism in the \Condor{negotiator} daemon allowed 
submitter records to be deleted,
if the submitter's priority factor was explicitly set and
the value was equal to that defined by \MacroNI{DEFAULT\_PRIO\_FACTOR}.
\Ticket{2442}

\item Fixed CPU detection on Windows, such that the correct number of CPUs
is detected when there are more than 32 CPUs.
\Ticket{2381}

\item Fixed a memory leak in the \Condor{negotiator},
caused by the failure to
free memory returned from some calls to \Procedure{param\_without\_default}.
\Ticket{2299}

\item Jobs run via \Prog{glexec} always had their \Env{PATH} environment
variable cleared.  Now, if \Env{PATH} was specified for the job, 
in any of the ways that job environment may be specified, 
this setting is used.
\Ticket{2096}

\item Fixed an infinite loop that could happen in Condor daemons
shortly after the receipt of a new connection.  
This problem was introduced in Condor version 7.5.6.
\Ticket{2413}

\item Fixed a problem in \Condor{hdfs} that caused it to exit shortly
after starting up,
if the configuration variables 
\MacroNI{HDFS\_DENY}, \MacroNI{HOSTDENY\_WRITE}, or \MacroNI{HOSTDENY\_READ} 
were not defined.
Previously, if \MacroNI{HDFS\_DENY} was
not defined, \MacroNI{HOSTDENY\_WRITE} and \MacroNI{HOSTDENY\_READ}
were used to build the deny list.  
Now \MacroNI{DENY\_WRITE} and \MacroNI{DENY\_READ} are also used.
\Ticket{2414}

\item Removed an extra copy of the java files required to run gt4 and gt42
grid universe jobs. This does not affect Condor's operation.
\Ticket{2435}

\item Fixed a problem that caused the \Condor{schedd} to crash when
writing to some user logs with specific names.  The specific names that
caused crashes are not easy to describe.
\Ticket{2439}

\item Fixed a bug in which the \Condor{schedd} failed to start up
when any job ClassAd attribute value contained the ASCII character 255.
\Ticket{2450}

\item Fixed a bug in which \Condor{preen} failed to honor the 
\Opt{-remove} option, and would always remove lock files.
\Ticket{2497}

\item \Condor{preen} expected an old format for local lock file paths;
it now understands the proper format.
\Ticket{2496}

\item \Condor{preen} would EXCEPT when processing multiple 
subdirectories for local locks.
\Ticket{2495}

\item In 32-bit Condor binaries, the \Attr{ImageSize} of processes larger than 
4 GBytes was reported as 4 GBytes.  This limit has been raised to 4095 GBytes.

\item \SubmitCmd{vm} universe jobs using Xen or KVM would fail to start,
if the disk image files were transferred from the submit machine
and the default value defined for \Macro{LIBVIRT\_XML\_SCRIPT} was used.
The script did not provide absolute path names for the files.
\Ticket{2511}

\item Fixed a bug in which a completed Xen or KVM \SubmitCmd{vm} universe 
job's modified disk image files would not be transferred back 
to the submit machine.
\Ticket{2530}

\item Fixed a bug in which a \Condor{starter} configured to use job hooks 
could fail to run a job, 
but not wait for the job exit hook to complete before exiting.  
The bug fix introduces the new configuration variable
\Macro{<Keyword>\_HOOK\_JOB\_EXIT\_TIMEOUT},
which defines the number of seconds the \Condor{starter} will wait
before continuing with a shut down,
if the job exit hook has not completed.
\Ticket{2543}

\item In Condor version 7.5.4, an improvement was made to avoid reliance on
the machine specified by \MacroNI{NEGOTIATOR\_HOST} 
matching a reverse DNS look up of the \Condor{negotiator}.
However, this improvement was not made to the dedicated scheduler,
so matchmaking of parallel jobs was still subject to the
problems associated with the old algorithm.  
Now, the dedicated scheduler benefits from the same improved algorithm as the
non-dedicated scheduler.
\Ticket{2540}
  
\item Occasionally there have been problems with Windows 
\Expr{LOGON\_NETWORK} authentication,
leading to users being locked out from their account.
The new configuration variable \MacroNI{SKIP\_WINDOWS\_LOGON\_NETWORK},
when set to \Expr{True},
fixes the problem by allowing this mechanism to be skipped entirely,
instead proceeding straight to the \Expr{LOGON\_INTERACTIVE} authentication. 
This bug only affected users using the \Condor{credd}. 
\Ticket{2549}  

\item Condor now correctly groups CREAM jobs based on how CREAM servers 
authorize and map them.
The servers map them based on X.509 proxy subject name 
and first VOMS attribute. 
Previously, all VOMS attributes were considered.
This could cause unexpected behavior due to the aliasing of CREAM leases
and proxy delegations.
\Ticket{2271}

\item Communication errors in the job lease renewal protocol were
sometimes not correctly handled.  This resulted in the job being
killed.
\Ticket{2563}

\end{itemize}

\noindent Known Bugs:

\begin{itemize}

\item None.

\end{itemize}

\noindent Additions and Changes to the Manual:

\begin{itemize}

\item The manual now contains a manual page for \Condor{rmdir},
a Windows only replacement for the built-in Windows \Prog{rmdir}
introduced in Condor version 7.6.0.

\end{itemize}


%%%%%%%%%%%%%%%%%%%%%%%%%%%%%%%%%%%%%%%%%%%%%%%%%%%%%%%%%%%%%%%%%%%%%%
\subsection*{\label{sec:New-7-6-3}Version 7.6.3}
%%%%%%%%%%%%%%%%%%%%%%%%%%%%%%%%%%%%%%%%%%%%%%%%%%%%%%%%%%%%%%%%%%%%%%

\noindent Release Notes:

\begin{itemize}

\item Condor version 7.6.3 released on August 23, 2011.

\end{itemize}


\noindent New Features:

\begin{itemize}

\item None.

\end{itemize}

\noindent Configuration Variable and ClassAd Attribute Additions and Changes:

\begin{itemize}

\item None.

\end{itemize}

\noindent Bugs Fixed:

\begin{itemize}

\item Fixed a bug causing parallel universe jobs to be preempted upon 
renewal of the job lease, 
which by default happens within 20 minutes. 
This meant that essentially no parallel universe job that took
longer than 20 minutes would ever finish.
\Ticket{2317}

\item When the specified job requirements expression contained a
reference to \Attr{RequestMemory}, there was inconsistent behavior:
in some cases the default \Attr{RequestMemory} requirements were
suppressed, and in other cases not.  Now, the default
\Attr{RequestMemory} requirements are always suppressed when there
are explicit references to \Attr{RequestMemory} in the job
requirements.

\item Fixed a bug that could cause Condor to crash when using 
the Local Credential Mapping Service (LCMAPS) with
GSI authentication.
\Ticket{2340}

\item Fixed a bug that caused the \Condor{collector} daemon to crash
upon receiving a CCB command,
when \Macro{ENABLE\_CCB\_SERVER} was changed from \Expr{True} to \Expr{False}
without restarting the daemon.
\Ticket{2357}

\item The GT2 GAHP no longer consumes all of the CPU when compiled
with threaded Globus libraries.
\Ticket{2345}

\item Fixed a problem introduced in Condor version 7.5.6, 
which led to local lock files for user log locking always being created 
whether or 
not \MacroNI{ENABLE\_USERLOG\_LOCKING} was set to \Expr{False}.
\Ticket{2116}

\item Installation as a service by the MSI installer on Windows platforms 
now sets the default of Automatic Delayed.
\Ticket{2318}

\item In PrivSep mode, if started as \Login{root}, 
the \Condor{master} re-executes itself as the \Login{condor} user.
Previously, supplementary groups were preserved.
Now supplementary groups are cleared and set to the list of groups
to which the \Login{condor} user belongs.
\Ticket{2376}

% commit 3d145180fd55b0d50e74656765cebe561c840830
% commit fea686687f5dda08908e03b5595c517b3ddda03a
\item Fixed a bug where setting \Macro{DAGMAN\_PROHIBIT\_MULTI\_JOBS} to
\Expr{True} caused SUBDAGs to stop working.
\Ticket{2331}

\item Fixed a bug that caused scheduler universe jobs submitted via
Condor-C or \Condor{submit} \Opt{-spool} to be held and be unable to run.
The hold reason given was \Expr{File <filename> is missing or not executable}.
\Ticket{2396}

\item \Condor{submit} now exits with an error,
if the command \Expr{hold = True} is in the submit description file
when using \Opt{-spool} or \Opt{-remote} as command-line arguments. 
This combination of settings resulted in jobs being unable to run.
\Ticket{2398}

\end{itemize}

\noindent Known Bugs:

\begin{itemize}

\item None.

\end{itemize}

\noindent Additions and Changes to the Manual:

\begin{itemize}

\item None.

\end{itemize}


%%%%%%%%%%%%%%%%%%%%%%%%%%%%%%%%%%%%%%%%%%%%%%%%%%%%%%%%%%%%%%%%%%%%%%
\subsection*{\label{sec:New-7-6-2}Version 7.6.2}
%%%%%%%%%%%%%%%%%%%%%%%%%%%%%%%%%%%%%%%%%%%%%%%%%%%%%%%%%%%%%%%%%%%%%%

\noindent Release Notes:

\begin{itemize}

\item Condor version 7.6.2 released on July 19, 2011.

\end{itemize}


\noindent New Features:

\begin{itemize}

\item Improved how \Condor{dagman} deals with certain parse errors:
missing node name or submit description file in JOB lines.
Also, \Condor{dagman}
now prints DAG input file lines as they are parsed, 
if the debug verbosity setting is 6 or above,
as set with the \Condor{submit\_dag} command line option \Opt{-debug}.

\end{itemize}

\noindent Configuration Variable and ClassAd Attribute Additions and Changes:

\begin{itemize}

\item None.

\end{itemize}

\noindent Bugs Fixed:

\begin{itemize}

% gittrac #2275 
\item Fixed a bug in the \Condor{negotiator} that impacted the processing 
of machine \MacroNI{RANK} such that \Condor{startd} \MacroNI{RANK}
preemption only occurred if the preempting user had sufficient user priority 
to claim another machine. 

% gittrac #2235 
\item \Condor{ssh\_to\_job} did not work on systems using the 
dash shell for \Prog{/bin/sh}.

% gittrac #2263 
\item \Condor{ssh\_to\_job} now works with jobs that are run via 
\Prog{glexec}. Previously, it did not. 

% gittrac #1642 
\item When \Prog{glexec} was configured with \Expr{linger=on},
the \Condor{starter} would become unresponsive for the duration of the job. 
For jobs longer than the value set by configuration variable
\MacroNI{NOT\_RESPONDING\_TIMEOUT},
this caused the job to be aborted. 
This also prevented job resource usage monitoring from working 
while the job was running.

% gittrac #2262 
\item Fixed a bug in hierarchical group quotas that caused 
a warning to be logged, despite correct implementation.

% gittrac #2261 
\item \Condor{preen} now properly respects the convention that
the \Opt{-debug} option causes \Procedure{dprintf} logging to \Code{stderr}. 

% gittrac #2253 
% gittrac #2294 
\item Fixed a problem introduced in Condor version 7.5.5 
that could cause the \Condor{schedd} to crash when a job was removed 
during negotiation or when an idle parallel universe job left the queue. 

% gittrac #2247 
\item Fixed a problem that sometimes caused the \Condor{procd} to die.
The chain of events for this fixed bug were that
the \Condor{startd} killed the \Condor{starter} due to unresponsiveness,
and the \Condor{procd} would die.
Then \Condor{startd} logged the message
\Expr{ProcD has failed} and the \Condor{startd} exited. 

% gittrac #2233 
\item Fixed a problem introduced in Condor version 7.6.1 
that caused the \Condor{shadow} to crash without successfully putting the job 
on hold when the user log could not be opened for writing. 

% gittrac #2210 
\item \Condor{history} no longer crashes when given a constraint expression 
longer than 512 characters. 

% gittrac #2248 
\item PBS and LSF grid jobs that arrive in a queue via Condor-C
or remote submission again work properly. 

% gittrac #2210 
\item Fix a bug that can cause the \Condor{gridmanager} to crash 
when a CREAM job ClassAd is missing the \Attr{X509UserProxy} attribute. 

% gittrac #2202 
\item Fix a bug that caused CREAM jobs to have incomplete input files,
if the \Condor{gridmanager} crashed during transfer of those input files. 

% gittrac #2201 
\item Fix a bug in \Condor{submit} that caused grid jobs intended for 
CREAM services whose names contain extra dashes to become held. 

\item Fixed a bug in which \Condor{submit} would try, 
but fail to open the Deltacloud password file,
when the file name was dependent on an expression specified with \Expr{\$\$()}.

% gittrac #2173 
\item If the \Attr{Owner} attribute was not set in the ClassAd associated
with a cluster of jobs,
shared spooled executables were not correctly cleaned up.

% gittrac #2238 
\item Fixed a bug for 64-bit versions of Windows in which the
user \Login{condor-reuse-slot<N>} showed up on the login screen.

% gittrac #2288 
\item Fixed a bug introduced in Condor version 7.5.5,
which changed the default value of the configuration variable
\Macro{INVALID\_LOG\_FILES} from the empty set to a file called \File{core}.
This resulted in core files being removed unexpectedly by \Condor{preen},
and that complicated debugging of Condor.
Previous behavior has been restored.

% gittrac #2278 
\item Fixed a bug existing since Condor version 7.5.5 on Windows platforms.
The installer installed Java jar files in the correct \verb|$(BIN)| directory,
while the value for the configuration variable 
\MacroNI{JAVA\_CLASSPATH\_DEFAULT} utilized the obsolete \verb|$(LIB)|
directory.
The installer now correctly sets \MacroNI{JAVA\_CLASSPATH\_DEFAULT} 
to the \verb|$(BIN)| directory.

% gittrac #2308
\item Fixed a problem causing Condor to fail to start when
privsep was enabled and the environment had any variables
containing newlines.

\end{itemize}

\noindent Known Bugs:

\begin{itemize}

\item For Condor versions 7.6.2, 7.6.1, and 7.6.0,
a bug causes parallel universe jobs to be preempted upon 
renewal of the job lease, 
which by default will happen within 20 minutes, 
essentially meaning that no parallel universe job that takes
longer than 20 minutes can ever finish.
The work around for this bug is to place the following
configuration variable in the configuration of the submit machine:
\begin{verbatim}
  STARTD_SENDS_ALIVES = FALSE
\end{verbatim}
A \Condor{reconfig} is required, 
after which the preempted parallel universe jobs will then be
able to run to completion.

\end{itemize}

\noindent Additions and Changes to the Manual:

\begin{itemize}

\item None.

\end{itemize}


%%%%%%%%%%%%%%%%%%%%%%%%%%%%%%%%%%%%%%%%%%%%%%%%%%%%%%%%%%%%%%%%%%%%%%
\subsection*{\label{sec:New-7-6-1}Version 7.6.1}
%%%%%%%%%%%%%%%%%%%%%%%%%%%%%%%%%%%%%%%%%%%%%%%%%%%%%%%%%%%%%%%%%%%%%%

\noindent Release Notes:

\begin{itemize}

\item Condor version 7.6.1 released on June 3, 2011.

\end{itemize}


\noindent New Features:

\begin{itemize}

\item None.

\end{itemize}

\noindent Configuration Variable and ClassAd Attribute Additions and Changes:

\begin{itemize}

\item None.

\end{itemize}

\noindent Bugs Fixed:

\begin{itemize}

% gittrac #2170 
\item A bug introduced in Condor version 7.5.5 caused the \Condor{schedd}
to die when its attempt to claim a slot for a parallel universe job 
was rejected by the \Condor{startd}. 

% gittrac #2059
\item \Condor{q} \Opt{-analyze} failed to provide detailed analysis of
the job's requirements expression when the expression contained ClassAd
function calls in some cases. 

% gittrac #2192
\item Fixed an incorrect exit code from \Condor{q} 
when invoked with the \Opt{-name} option and using Quill.

%gittrac #2013
\item Fixed a segmentation fault bug introduced in Condor version 7.5.5,
in the checkpoint and restart of jobs using compressed checkpoint images
under the standard universe.
By default, Condor will not compress checkpoints under the standard universe.
Jobs which do not compress their checkpoints were immune to this bug.  
Compressed checkpoints are only available in 32-bit versions of Condor.
Generation of checkpoints in 64-bit versions of Condor are unaffected.

% gittrac #2069
\item In Condor version 7.6.0, the \Condor{schedd} would create a 
spool directory for every job. The corrected and previous behavior 
has now been restored, 
which is to create a spool directory only when needed.

%gittrac #2086
\item Fixed a bug introduced in Condor version 7.5.2,
that caused the \Condor{negotiator} daemon to crash
if any machine ClassAds contained cyclical attribute references.

%gittrac #2101
\item Fixed a bug that caused usage by \SubmitCmd{nice\_user} jobs to
be charged to the user directly rather than `nice-user.\emph{user}'.
This bug was introduced in the 7.5 series.

%gittrac #2081
\item Fixed bugs in the RPM init script that could cause some 
shutdown failures to be unreported, 
and they could cause status requests,
such as \Expr{service condor status},
to always report Condor as inactive.

\item Fixed a bug in the \Condor{gridmanager} that could cause a crash 
when a grid type \SubmitCmd{amazon} job was missing required attributes.

%gittrac #2105
\item Fixed bug in the \Condor{shadow}, in which it would treat 
the closed socket to the execute machine as an error,
after both it had asked for the claim to be deactivated and the 
\Condor{schedd} daemon was busy.  
As a result, a busy \Condor{schedd} could result in the job being re-run.

%gittrac #2109
\item The matchmaking attributes 
\Attr{SubmitterUserResourcesInUse} and \Attr{RemoteUserResourcesInUse} 
no longer ignore \Attr{SlotWeight}, if used by the \Condor{negotiator}.

%gittrac #2102
\item On Windows, the \Condor{kbdd} daemon was missing changes to the
port on which the \Condor{startd} was listening.
This resulted in failure of the \Condor{kbdd} to send updates in 
keyboard and mouse activity,
further causing the failure of policy implementation that relied upon 
knowledge of the activity.

%gittrac #2163
\item Fixed a bug present throughout ClassAds,
in which expressions expecting a floating point value returned an error,
if the expression actually evaluated to a boolean.
This is most common in machine \MacroNI{RANK} expressions.

%gittrac #2172
\item Fixed a bug in the \Condor{negotiator} daemon,
which caused a crash if the \Condor{negotiator} was reconfigured 
during a negotiation cycle, 
but only if hierarchical group quotas were in use.

%gittrac #2162
\item Fixed a bug in which when submitting a job into the \Condor{schedd}
remotely, or with spooling, 
the file transfer plug-ins activated on the submit machine 
and pulled down all the specified URLs in the transfer list 
to the spool directory. 
This behavior has been changed so that URLs are only downloaded 
when the job is actually running with a \Condor{starter} above it. 
This is usually on an execute node, but could also be in the local universe. 

%gittrac #2178
\item Removed the requirement that the Condor GAHP needs DAEMON-level 
authorization access to the \Condor{gridmanager}. 

%gittrac #2181
\item On Windows platforms only, 
fixed a bug that could cause a sporadic access violation 
when a Condor daemon spawned another process.

%gittrac #2191
\item Fixed a bug that would cause the \Condor{startd} to 
incorrectly report \Expr{Benchmarking} as its activity, instead of \Expr{Idle}
when there was a problem launching the benchmarking programs. 

%gittrac #2193
\item Fixed a bug in which the \Condor{startd} can get stuck in a loop,
trying to execute an invalid, non-existent Daemon ClassAd Hook job. 

%gittrac #2088
\item Fixed a bug in which the dedicated scheduler did not correctly 
deactivate claims,
tending to result in jobs that appear to move back and forth between
the \Expr{Idle} and \Expr{Running} states,
with the \Condor{shadow} daemon exiting each time with status 108.

\end{itemize}

\noindent Known Bugs:

\begin{itemize}

\item None.

\end{itemize}

\noindent Additions and Changes to the Manual:

\begin{itemize}

\item None.

\end{itemize}


%%%%%%%%%%%%%%%%%%%%%%%%%%%%%%%%%%%%%%%%%%%%%%%%%%%%%%%%%%%%%%%%%%%%%%
\subsection*{\label{sec:New-7-6-0}Version 7.6.0}
%%%%%%%%%%%%%%%%%%%%%%%%%%%%%%%%%%%%%%%%%%%%%%%%%%%%%%%%%%%%%%%%%%%%%%

\noindent Release Notes:

\begin{itemize}

\item Condor version 7.6.0 released on April 19, 2011.

% gittrac #2016
\item Prior to Condor version 7.5.0, commenting out \MacroNI{PREEN} in the
  default configuration file disabled \Condor{preen}.  
  Starting in Condor version 7.5.0,
  there was an internal default value for \MacroNI{PREEN}, so if
  the configuration variable was not set in any configuration file,
  \Condor{preen} would still run.
  To disable this functionality, \MacroNI{PREEN} can be explicitly set to
  nothing.

\end{itemize}


\noindent New Features:

\begin{itemize}

\item Condor can now create and manage virtual machine instances in a
cloud service via Deltacloud. This is done via the new
\SubmitCmd{deltacloud} grid type in the grid universe.
See section ~\ref{sec:Deltacloud} for details.

% gittrac #1931
\item Improved scalability of submission of cream grid type jobs.

\end{itemize}

\noindent Configuration Variable and ClassAd Attribute Additions and Changes:

\begin{itemize}

\item The new configuration variable \Macro{DELTACLOUD\_GAHP} specifies
where the \Prog{deltacloud\_gahp} binary is located. This binary is used to
manage deltacloud grid type jobs in the grid universe.
In a normal Condor installation, the value should be
\File{\$(SBIN)/deltacloud\_gahp}.

\item Several new job ClassAd attributes have been added to support
the deltacloud grid type in the grid universe.
These attributes are taken from the submit description file:
\AdAttr{DeltacloudUsername},
\AdAttr{DeltacloudPasswordFile},
\AdAttr{DeltacloudImageId},
\AdAttr{DeltacloudRealmId},
\AdAttr{DeltacloudHardwareProfile},
\AdAttr{DeltacloudHardwareProfileCpu},
\AdAttr{DeltacloudHardwareProfileMemory},
\AdAttr{DeltacloudHardwareProfileStorage},
\AdAttr{DeltacloudKeyname}, and
\AdAttr{DeltacloudUserData}.
%\AdAttr{DeltacloudRetryTimeout},
These attributes are set by Condor when the instance runs:
\AdAttr{DeltacloudAvailableActions},
\AdAttr{DeltacloudPrivateNetworkAddresses},
\AdAttr{DeltacloudPublicNetworkAddresses}.
See section ~\ref{sec:Deltacloud} for details of running jobs under
Deltacloud, and see section ~\ref{sec:Job-ClassAd-Attributes}
for definitions of these job ClassAd attributes.

% gittrac #2024
\item The configuration variable \Macro{JAVA\_MAXHEAP\_ARGUMENT} 
  has been removed. 
  This means that Java universe jobs will now run with the JVM's 
  default maximum heap setting,
  unless specified otherwise by the administrator using the configuration
  of \Macro{JAVA\_EXTRA\_ARGUMENTS},
  or by the job via 
  \SubmitCmd{java\_vm\_args} in the submit description file
  as described in section~\ref{sec:Java}.

% gittrac #2066
\item The configuration variable \Macro{TRUST\_UID\_DOMAIN}
  was set to \Expr{True} in the default \File{condor\_config.local}
  in the rpm and Debian packages.  This is no longer the case.
  This setting will therefore use the default value \Expr{False}.

\item The configuration variable \Macro{NEGOTIATOR\_INTERVAL} was set
  to 20 in the default \File{condor\_config.local} in the rpm and
  Debian packages.  This is no longer the case.  This setting
  therefore will use the default value 60.

\end{itemize}

\noindent Bugs Fixed:

\begin{itemize}

% gittrac #1957
\item Fixed a bug in \Condor{dagman} that caused it to fail when in recovery
mode in the face of a specific combination of node job failures with
retries.

% gittrac #1991
\item Fixed a bug that resulted in the spooled user log not being
  fetched by \Condor{transfer\_data}.  Prior to Condor version 7.5.4, this
  problem affected spooled jobs submitted with an explicit list of
  output files to transfer.  In Condor version 7.5.4, this problem also
  affected spooled jobs that auto-detected output files.

% gittrac #1985
\item When a job is submitted with the \Opt{-spool} option to \Condor{submit},
the \Condor{schedd} now correctly writes the submit event to the user log 
in its spool directory. 
Previously, it would try to write the event using the user
log path given to \Condor{submit} by the user, 
which \Condor{submit} may not have access to.

% gittrac #2001
\item Fixed a file descriptor leak in the \Condor{vm-gahp}. The leak would
cause the daemon to fail if a VMware job ran for more than 16 hours on a
Linux machine.

%gittrac #2017
\item Fixed a bug in \Condor{dagman} that caused it to treat any dollar
sign in the log file name of a node job's submit description file
as an illegal \Condor{dagman} macro.
Now only the sequence of characters \Expr{\$(} delimits a macro.

\end{itemize}

\noindent Known Bugs:

\begin{itemize}

\item There are two known issues related to the automatic creation
of checkpoints with the Condor checkpointing library in 
Condor version 7.6.0.
The first is that compression of
standalone checkpoints is disabled for 32-bit binaries.
It was always disabled previously, for 64-bit binaries.
A standalone checkpoint is one that is run outside
of Condor's standard universe.  The second problem has to do with compressed
32-bit checkpoint files within the standard universe.
If a user requests a compressed 32-bit checkpoint in the standard universe,
the resulting checkpoint will not be compressed.
As with standalone checkpoints, this has never been supported
in 64-bit binaries.  We hope to fix both problems in 
Condor version 7.6.1.

\end{itemize}

\noindent Additions and Changes to the Manual:

\begin{itemize}

\item None.

\end{itemize}


% as of April 2012, Karen no longer wants to include these older
% version histories with the 7.4 and 7.5 manuals.
%\input{version-history/7-5.history.tex}
%\input{version-history/7-4.history.tex}
% as of April 2011, Karen no longer wants to include these older
% version histories with the 7.6 and beyond manuals.
%\input{version-history/7-3.history.tex}
%\input{version-history/7-2.history.tex}
%\input{version-history/7-1.history.tex}
%\input{version-history/7-0.history.tex}
% Oct 2009, as we release 7.4, Karen commented out inclusion of the
% 6.9 and 6.8 histories
%\input{version-history/6-9.history.tex}
%\input{version-history/6-8.history.tex}
% Dec 2007, as we release 7.x, Karen commented out the older histories
%\input{version-history/6-7.history.tex}
%\input{version-history/6-6.history.tex}
% Feb 2007 -- still in the manual source, just not incorporating
% these old histories into the finished product, thereby
% reducing the size of the manual by 200 pages
%\input{version-history/6-5.history.tex}
%\input{version-history/6-4.history.tex}
%\input{version-history/6-3.history.tex}
%\input{version-history/6-2.history.tex}
%\input{version-history/6-1.history.tex}
%\input{version-history/6-0.history.tex}
